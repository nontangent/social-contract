\chapter{序論}
  \section{本研究の動機}
  \label{motivation}
  % 社会契約をひとことで
  社会契約とは、ある集団においてその成員達が合意された約束を必ず履行する状態に至るプロセスである。
  % 中世の社会契約論
  中世ヨーロッパにおいて、この概念はホッブズやロック、ルソーによって議論され
  「国家において、なぜ国民が法に従うのか」という問いに理論的な根拠をもたらすことで王権神授説を否定する主張として注目された。
  % 現代の社会契約論
  現代においては、ロールズ\cite{rawls1971}やハーサニー\cite{harsanyi1955}が中世の社会契約論の一般化を試み、
  そこから導きだされる公正としての正義について議論した。
  それ以降、社会契約は正義や倫理、道徳といった概念と結びつけて議論されることになる。
  % 時代の変化とともに議論される意義も変わる社会契約
  社会契約の研究の意義は社会の変化とともに移り変わっており、
  社会の動向の変化に伴って今後も様々な意義が生まれるだろう。
  その一つの例として、我々はインターネットのもたらす社会の変化が社会契約の研究に新たな意義を生じさせると考えている。

  % インターネットのグローバル化
  2020年のGSMA\cite{gsma2020}の調査では、モバイル回線のユニークな契約者数は5.8億人に登り、
  世界人口の70\%がモバイル回線を所有していることが示唆されている。
  インターネットユーザーが全世界で増加する一方、
  Google\cite{google}やFacebook\cite{facebook}のようなサービスは全世界にユーザーを抱えるまで成長し、
  それらの運営会社も多国籍企業として多数の国に支社を置くようになっている。
  こうした企業のガバナンスは、もはや特定の国家の法によって管理するのは困難なレベルまで達している。
  % ブロックチェーンの出現
  また、誰にも送金を止めさせないことを目的としたBitcoin\cite{nakamoto2008bitcoin}の登場を皮切りに、
  ブロックチェーンを用いた様々な分散型台帳技術を用いたサービスが出現している。
  こうしたサービスはP2Pと呼ばれる通信技術を用いており、
  世界中のコンピューターがそのノードとしてシステムの運用を担っていため、
  特定の国家がそのサービスを規制することは極めて困難である。
  % これからのインターネット
  グローバル化や新技術の登場は今後も止まることなく、インターネットは国家という枠組みを超越した社会インフラへ進化していくだろう。
  「地球規模OS」\cite{saito&ikemoto2008} のような地球規模で資源を抽象化して共有可能にするプラットフォームが当たり前に存在する未来がやってくるかもしれない。

  % 今までの社会契約の議論じゃだめ
  % これまで以上に厳密なモデリングが必要
  そうした未来に向かって必ず衝突するのは、
  国家を超越したインターネット上に存在するサービスの正当性をどのように保証するかという問題である。
  先に述べたとおり、特定の国家の法によってサービスの運営母体を規制しサービスの正当性を保証させることは困難になりつつある。
  また、Bitcoinブロックチェーンにおいては、Proof of Workと呼ばれる技術によって確率的にその正当性を保証しようとしているが、
  そのためだけに世界中で大量の計算リソースが消費され続けているのが現実である。
  もちろんそれ以外に手段が存在し得ない可能性もあるものの、
  これが国家を超越した社会インフラの正当性を保証するためにベストな方法であるとは容易に納得し難い。

  % インターネットで社会契約
  我々はこの問題を解決する糸口は社会契約の理論研究にあると考えている。
  かつて中世の社会契約論者が説明したように、国家の法が国民の社会契約によって成立するのであれば、
  国家を超越したインターネット上の法はインターネットユーザーによる社会契約によって成立するのではないだろうか。
  仮に国家を超越したインターネット上の法が成立するのならば、それによってサービスの正当性を保証すればよい。
  こうしたアイディアが本研究が社会契約を取り扱うモチベーションである。

  \section{本研究の貢献}
  本研究では、我々はインターネットユーザーの集団を想定して社会契約の理論を再構築する。
  当然のごとく、彼らはコンピューターを用いてインターネットでやり取りをするため、
  社会契約のモデルは(コンピューターで)計算可能なレベルまで抽象化されている必要がある。
  そこに至る最大の障壁は強制執行力と呼ばれるプレイヤーの利得の分配を強制執行する力の存在である。
  これまでの社会契約の研究においては、この強制執行力が成員のどのような振る舞いによって生じるのか明確にできていないため、
  社会契約全体を計算可能なレベルまで抽象化することは困難であった。
  我々はこの強制執行力を自己組織化させる成員達の振る舞いを示すことで、
  社会契約全体を計算可能なレベルでの抽象化すること示す。
  また、マルチエージェントシミュレーションによってそのモデルを検証することで、
  成員の性質と社会契約の成立の関係性を明らかにする。

  \section{本論の構成}
  本論の構成は次の通りである。
  第2章では本論を読みすすめるにあたって必要となる前提知識と先行研究について述べる。
  第3章では本研究で取り扱う問題について詳細に定義する。
  第4章では先に定義した問いに対する我々の仮説とそれを示すために必要となる3つの補題について述べるとともに、それらの全体的な検証方法について述べる。
  続く3つの章では、それぞれの補題についての具体的な検証方法と検証結果について述べる。
  第8章では全ての検証を通してわかったことについてまとめるとともに、そこから生まれた新たな疑問と今後の研究アイディアについて述べる。
