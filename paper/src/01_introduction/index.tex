\chapter{序論}

  % \section{なぜ私たちは法に従うのか?} %導入
  % % この問いに答えることが社会契約という概念が研究される主な目的である。
  
  % % 法が守られる世界を前提に生きている
  % 普段、我々は、人々が法に従うことを当然のように信じており、
  % この問いについて深く考えることはめったにない。
  % 例えば、あなたが近所のお店でコーヒーを注文したとしよう。
  % あなたはメニューを見て、注文を行い、代金を支払って、コーヒーを受け取る。
  % 結果的に、あなたはコーヒーを手にして、店主は代金を手にする。
  % 日常的に経験するであろう法の守られた世界である。

  % % 法が守られない世界も存在する
  % しかし、仮に、あなたがメニューを見て、注文を行い、請求された金額を支払ったのに、店主が飲み物を渡さなかったとしよう。
  % あなたは次にどうするだろうか。警察に通報するだろうか、裁判所に訴えて返金を要求するだろうか。
  % もし警察に通報したり、裁判所に訴えたりしたとして、彼らと店主が共謀していて、不正をなかったことにされたらどうするだろうか。
  
  % % 現実的に後者の世界である可能性を捨てきれないが、前者の世界であると我々は信じている
  % 我々は無意識にどこかで、店主を信頼している。そうでなくとも、店主が不正を働いた場合に、警察が解決してくれることを信頼している。
  % 仮に店主と警察の間で賄賂が渡されており、両者が共謀していたとしても、裁判所が正常に機能していることを信頼している。
  % 裁判所が機能しなくても不正を行ったものが他の何かを罰してくれることを信じている。

  % 抽象的に言い換える
  % つまるところ、我々は、不正が起きても当事者でない強制力をもった機関(外部の強制執行機関)がその不正を正してくれることを信じている。
  % そうでなくとも、その機関が不正を働いても、さらなる外部の強制執行機関がその不正を正してくることを信じている。
  % そして、我々は暗黙のうちにどこかでこの信頼の連鎖の根源を考えることをやめてしまう。

  % なぜ店主は信頼できるのか?なぜ警察は信頼できるのか?なぜ裁判所は信頼できるのか?なぜ国家は信頼できるのか?
  % なぜ法律は信頼できるのか?なぜ立法機関は信頼できるのか?なぜ国会議員は信頼できるのか?なぜ国民は信頼できるのか?

  % この信頼の連鎖の根源を探ることが、「なぜ私たちは法に従うのか?」という問いに、理論的な答えを導き出すことに繋がるだろう。
  

  % \section{これまでの社会契約の議論} %先行研究
  % これまで、この問いは社会契約という領域で研究されてきた。
  % 近代において、はホッブズやロック、ルソーによって、民主国家の正当性を示すために議論され、
  % 王権神授説に対する反論としてフランス革命の理論的な石杖となった。
  % 現代においては、ロールズやハーサニーが彼らの社会契約を一般化したモデルを構築し、
  % 「公正な正義とはいかなるものか?」という問いへの議論に発展した。

  % % 社会契約の議論の進め方
  % 社会契約は議論は伝統的に、自然状態と社会状態の2つの状態を仮定することから始まる。
  % 自然状態とは、ある集団において、その成員が法に従うことに強制力の働かない状態であり、
  % 社会状態とは、全ての成員が法に従うことに強制力の働く状態である。
  % この自然状態から社会状態へと遷移するプロセスを明らかにすることで、
  % 法に従うことに強制力が働くメカニズムの解明を試みてきた。

  % % 
  % 最近では、このメカニズムをゲーム理論や進化ゲーム理論を用いて解析した研究も増えており、
  % 「道徳とは何か?」、
  % 「どのような場合に我々は法に従うのか?」、
  % 「どういったプロセスで法に従うのか」というより派生した問いに主眼が置かれている。


  \section{本研究の動機}
  % 社会契約をひとことで
  社会契約とは、ある集団において、その成員達に合意された約束を遵守させる強制力が生じるプロセスである。
  % 中世の社会契約論
  中世において、社会契約はホッブズやロック、ルソーによって議論され、
  なぜ国家において国民が法に従うのかを論理的に説明しようと試みることで
  王権神授説を否定する主張として注目された。
  % 現代の社会契約論
  現代においては、ロールズやハーサニーが中世の社会契約論の一般化を試み、
  その社会契約のプロセスの果に導きだされる公正な正義とはどういったものかを議論した。
  それ以降、社会契約は倫理や正義論といった概念と結びつけて考えられることが多くなった。
  
  % 時代の変化とともに議論される意義も変わる社会契約
  このように社会契約が研究される意義は、社会の変化とともに移り変わっている。
  今後も社会の様々な変化に伴って社会契約の研究意義は変化していくだろう。
  その中でも、インターネットの普及がもたらす社会の変化は、社会契約の研究に新たな意義を与えると考えている。

  % インターネットのグローバル化
  2020年のGSMA\cite{gsma2020}の調査では、モバイル回線のユニークな契約者数は5.8億人に登り、
  実に世界人口の70\%がモバイル回線を所有していることが示唆されている。
  インターネットユーザーが全世界で増加する一方、
  Google\cite{google}やFacebook\cite{facebook}のように全世界にユーザーをかかえるまで成長するサービスが現れ、
  そうしたサービスの運営会社も多国籍企業として多数の国に支社を置くようになっている。
  こうした企業のガバナンスは、もはや一国の法にのみ依存するものではなくなっている。

  % ブロックチェーンの出現
  また、誰にも送金を止めさせないことを目的としたBitcoin\cite{nakamoto2008bitcoin}の登場を皮切りに、
  ブロックチェーンを用いた様々な分散型台帳技術を用いたサービスが出現している。
  こうしたシステムはP2Pと呼ばれる通信技術を用いており、
  世界中のコンピューターがそのノードとしてシステムの運用を担っていため、
  単一の国家がそのサービスを規制するのは極めて困難である。
  
  % これからのインターネット
  こうしたインターネット上のサービスのグローバル化や新技術の登場により、
  インターネットを国家という枠組みを超越した社会インフラへ進化していると言えるだろう。
  こうした進化の先に、「地球規模OS」\cite{saito&ikemoto2008} のような
  地球規模で資源を抽象化して共有可能にするシステムが当たり前に存在する未来が到来すると期待される。

  % 今までの社会契約の議論じゃだめ
  % これまで以上に厳密なモデリングが必要
  そうした未来の実現に向かって必ず衝突する問題は、
  国家を超越したインターネット上に存在するサービスの正当性をどのように保証するかという問題である。
  先に述べたとおり、一国の法による拘束力では、サービスの運営母体を規制することは困難である。
  Bitcoinブロックチェーンにおいては、Proof of Workと呼ばれる技術によって、
  確率的にその正当性を保証しようとしているが、
  そのためだけに世界中で大量の計算リソースが消費され続けている。
  これが国家を超越した社会インフラを維持するベストな方法だとは安易に納得したくない。

  % インターネットで社会契約
  我々はこの問題を解決する糸口は社会契約の理論研究にあると考えている。
  かつて中世の社会契約論者が説明したように、国家の法が国民の社会契約によって成立するのであれば、
  国家を超越したインターネット上の法はインターネットユーザーによる社会契約によって成立するのではないだろうか。
  仮に国家を超越したインターネット上の法が成立するのならば、それによってサービスの正当性を保証すればよい。
  こうしたアイディアが本研究のモチベーションである。

  \section{本研究の貢献}
  本研究では、インターネットユーザーの集団を想定して社会契約の理論を再構築する。
  当然のごとく、彼らはコンピューターを用いてインターネットでやり取りをするため、
  社会契約のモデルは(コンピューターで)計算可能なレベルまで抽象化する。
  それにあたって課題となるのは、強制執行力である。
  これまでの研究において、この強制執行力の存在は暗黙的に存在が仮定されているか、外部の強制執行力としてその存在が仮定されているかのいづれかであった。
  Binmoreの研究では、この外部の強制執行力が存在しない場合についても社会契約が成立することが示されている\cite{binmore2005}が、
  その説明は還元主義的なものであり、具体的に成員のどのような振る舞いによってそれがなし得るのか記述することはできなかった。
  本研究では、この内部的に強制執行力が生じるメカニズムを成員の振る舞いとして具体的に記述を可能にする。
  これにより、社会契約の全体を計算可能なレベルでのモデリングすることが可能となる。
  また、コンピューターシミュレーションを用いて、そのモデルを検証することで、
  ある集団における成員の性質と社会契約の成立の関係性を明らかにする。

  % \section{本論のアプローチ}
  % 本論では、ある集団において外部の強制執行力に依らずに取り決めが遵守されるのは、
  % 各成員の振る舞いによる複雑系の中で取り決めを遵守させるシステムが自己組織化されているためだと考え、
  % その複雑系の詳細を商取引ゲームというアイディアを基にモデリングを試みる。
  % 商取引ゲームとは、商取引システムという唯一の外部の強制執行力の存在を仮定した上で、商取引をモデリングした非協力戦略型ゲームである。
  % この商取引システムは、商取引契約が履行されたか否かの真の結果を観察することはできないが、
  % プレイヤーから報告された結果を観察することができ、
  % それに応じて各プレイヤーの通貨の保有量を操作することのできるシステムである。
  
  % 本論のアプローチは主に2つの段階に分けることができる。
  % 1つは、この商取引システムの通貨の保有量を操作する機能を用いて、
  % 商取引ゲームにおいて不正が生じないようなインセンティブ設計を行うことであり、
  % もう1つは、商取引契約の内容にプレイヤーの振る舞いを記述することで、
  % この商取引システムを複雑系の内部で自己組織化されたシステムとして設計することである。

  % この2つのアプローチが成功したとき、
  % 商取引契約の内容には、不正を防止可能なインセンティブ設計を行える商取引システムを〜が定義されており、
  % この商取引契約はその契約によって自己組織化された内部の商取引システムによって履行されるようになるはずである。

  % ここで商取引契約の内容とは先に定義した社会契約における取り決めであり、プレイヤーとは同様にある集団の成員であるため、
  % ここで本論の目的である、社会契約を一般化した計算可能なモデルの構築が果たされることとなる。

  % 一見、契約を履行させるシステムによってそのシステムの正当性を担保する契約を履行させるという論理は、
  % 循環をはらんでいるように思えるだろうが、実際のところはこの2つの時間軸に依存しあっている。それ故に初期値によってその社会契約が成立するか否かは全く異なるものとなり、その結果を全て把握することは困難である。
 
  % しかしながら、幸いなことにこの複雑系としての社会契約は計算可能なモデルであるため、そのカオスをコンピューターシミュレーションによって近似することができるだろう。

  % 原点に立ち返るなら「なぜ我々は法を守るのか」という疑問に対して、
  % 本論が提示できる答えは「他の人も法を守っているから」というものになるだろう。
  % では具体的にどういう性質の成員がどのくらいいれば法が守られるのか、
  % それが本論の提示する複雑系としての社会契約のモデルから明らかになるだろう。

  \section{本論の構成}
  % 本論では

  % 先に上げた2つのアプローチについて、
  % 前者の不正を防止する商取引システムのインセンティブ設計は3章と4章で、
  % 後者の商取引システムを自己組織化する商取引契約の内容の設計は5章で説明する。

  % 3章では利己的なプレイヤーのみで行われる商取引ゲームにおいて、
  % 報告された結果に基づいて通貨の保有量を操作しても、
  % 不正を防止することは不可能であることを示す。

  % 4章では、商取引に失敗した場合にその結果を正しくシステムに報告するプレイヤーを倫理あるプレイヤーと考し、
  % こうしたプレイヤーのみで行われる「倫理ある商取引ゲーム」について考える。
  % この限定合理性の上でインセンティブ設計を行うことで、
  % 倫理あるプレイヤー以外のプレイヤーが参加する商取引ゲームであっても、
  % 参加するプレイヤー達の性質によっては不正を防止することが可能であることを示す。

  % 5章では、
  % 商取引システムを運営する約束を、商取引ゲームで取引されていた商品(サービス)とすることで、
  % 外部の強制執行力として存在していた商取引システムを自己組織化させる。
  % これによって、外部の強制執行力に依存することなく、
  % 社会の成員たちによって構成された複雑系の内部で商取引システムが稼働し、
  % 「倫理ある商取引ゲーム」と同様に参加するプレイヤーの性質によって不正が起きなくなることを示す。
  
  % 6章では、
  % 本論では、ここまでの内容をまとめるとともに、
  % 今回紹介した複雑系としての社会契約のモデルの課題を上げ、今後の研究についての計画述べる。

