\chapter{複雑系としての社会契約}
「倫理ある商取引ゲーム」において、倫理という限定合理性を仮定した上で商取引システムの仕様を決定すれば、
プレイヤーの構成によっては不正が防止され、商取引契約の内容が果たされるようになることがわかった。
本章では、この商取引契約の内容に焦点を当て、これまで外部の強制執行力として存在が仮定されていた「商取引システム」を、
ある集団の成員が構成する複雑系の内部で自己組織化されたシステムとして再現する商取引契約の内容について提案し、
マルチエージェントシミュレーションによって、商取引システムが存在しない場合でも、社会契約が成立しうることを示す。

\section{提案手法}
外部の強制執行力として存在していた商取引システムを、各プレイヤーが構成する複雑系の内部で自己組織化されたシステムとして実現させる。
そのためには、各プレイヤーが商取引システムを所有してその役割を果たした上で、
それぞれのプレイヤーの所有する「倫理ある商取引ゲーム」の章で扱った商取引システムの仕様どおりに動作している必要がある。
商取引システムは各プレイヤーの初期の通貨の保有量と各時刻の商取引の記録から、
一意に現在の時刻の各プレイヤーの通貨の保有量を計算するシステムであるため、
商取引システムの動作が各プレイヤーで一致するには、初期の通貨の保有量と各時刻の商取引の記録が決定されていればよい。

そこで、各プレイヤーが記録している過去の商取引の結果を互いに確認し、
差異が生じた場合に不正を報告する商取引契約の内容を考案することで、
それを可能にする。具体的にが下記のような商取引契約の内容を提案する。

\subsection{「商取引システム」を自己組織化する商取引契約}
\begin{description}
  \item[step 1] 時刻tを0とする。
  \item[step 2] プレイヤーの人数と各プレイヤーが初期に保有する通貨の量を任意に決定する。
  \item[step 3] N人のプレイヤーが互いに全てのプレイヤーとsellerとbuyerの2つの役割で商取引ゲームを行う周期ある順序を決定する。この周期の長さはN*N-1となる。
  \item[steo 4] 時刻tを1進める。
  \item[step 5] step3で決定した順序にに基づいて時刻tのsellerとbuyerを決定する。
  \item[step 6] 自分が$seller$ならば、時刻$\max\{0, t-n*(n-1)\}$から時刻$t$までに報告された商取引ゲームの記録のうち、
    報告者がその商取引ゲームの記録のbuyerと一致するものを全てbuyerに送信する。
  \item[step 7] 自分が$buyer$ならば、$seller$から受け取った商取引ゲームの記録をメモする。
    この際、記録する商取引の記録の報告者を$seller$に書き換える。
  \item[step 8] 自分が$buyer$ならば、時刻$\max\{0, t-2*n*(n-1)\}$から時刻$\max{0, t-n*(n-1)}$までの各時刻について、
    メモされた商取引の記録のうち自身が報告者の記録はそのまま商取引システムに保存する。
    それ以外の記録は、「成功」と「失敗」のそれぞれの報告者について、自身の所有する商取引システムの通貨保有量の総和をとり、
    大きい方の結果を採用して商取引システムに保存する。
  \item[step 9] 自身が$buyer$ならば、支持する商取引の記録を自身の所有する商取引システムに保存する。
  \item[step 10] 自身が$buyer$ならば、時刻$\max\{0, t-2*n*(n-1)\}$から時刻$\max{0, t-n*(n-1)}$までの各時刻について、
    自身の所有する商取引システムに保存した商取引の記録と、報告者が$seller$であるメモされた商取引の記録を比較し、
    全てが一致するする場合には「成功」を1つでも一致しない場合は「失敗」を全てのプレイヤーに報告する。
  \item[step 11] 全てのプレイヤーは$buyer$から受けた商取引の記録をメモする。この際、商取引の記録の報告者を$buyer$に書き換える。
  \item[step 12] step4に戻る。
\end{description}

\section{実験方法}
下記の8種類のエージェントをランダムに8体用意する。
時刻$t=1120$まで下記の施行を繰り返し、各エージェントの保有する商取引システムが有効だと判断した過去56回の商取引について、
報告された成功率と真の成功率を記録する。
この施行を8000回づつ繰り返しサンプリングを行う。

\subsection{8種類のエージェント}
\begin{itemize}
  \item タイプA…
  \item タイプB…
  \item タイプC…
  \item タイプD…
  \item タイプE…
  \item タイプF…
  \item タイプG…
  \item タイプH…
\end{itemize}

\section{評価}
報告された成功率と真の成功率の両者が100\%になっている場合を社会契約が成功した場合とし、
タイプAのエージェントと社会契約の成功に達した割合をプロットする。

