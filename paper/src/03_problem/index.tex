\chapter{問題提起}
\section{本論の問題提起}
これまでのゲーム理論を用いた社会契約の研究では、
限定合理的な成員の繰り返しゲームによって道徳的な規範が創発され社会契約が成立することがわかっている\cite{skyrms1996}\cite{skyrms2004}。
また、社会契約が成立した際、どのような解が選択されるのかについても研究されてきた。\cite{rawls1971}\cite{harsanyi1955}\cite{gauthier1986}
しかしながら、こうした議論の中では強制執行力の存在が仮定されており、
社会契約全体をモデリングするにはこの力が集団の内部で生じるメカニズムを解明する必要がある。

これについてBinmoreの2005年の研究\cite{binmore2005}では、
「外部の強制執行力が存在しない場合、社会の過去の履歴が決定する社会指標を用いることで決定される平等主義的交渉解が選択される」という結論に至っているが、
この研究は還元主義的なアプローチを用いているため
社会契約に必要な履歴の定義や社会指標の計算方法、成員達の具体的な振る舞いを明確に記述できていない。
また、この研究はすべての成員が各成員の戦略の選好を共有知識として持つという仮定の上に成り立っているが、
こうした仮定は集団の人数が多い場合やインターネット上など、直接的に他の成員の行動を観察できない環境では現実的ではない。

そこで本研究では、集団の外部に強制執行力が存在しない場合に社会契約が成立しうるのかという問いに取り組み、
その問を解く過程で社会契約の成立に必要となる過去の履歴と社会指標の決定方法、成員達の振る舞いを計算可能なレベルで明確にする。
ここでの社会契約の成立とは、その集団の成員達が必ず合意された約束を履行する状態に至ることである。

% 本研究では、
% \subsection{仮定}
% \begin{enumerate}
%   \item 報告者($reporter$)以外の成員は約諾者($promisor$)の行動を観察できない。  
%   \item 各成員は同期された時刻を知ることができる。 
%   \item 各成員は集団の人数について事前に合意している。
%   \item 各成員は成員の社会指標の決定方法について事前に合意している。
%   \item 各成員は各時刻の約諾者と報告者について事前に合意している。
%   \item 送信されたすべてのメッセージは正しく到達する
%   \item メッセージの受信者は誰が送信したのかわかる
%   \item 各成員はメッセージが届かないことを検知できる
%   \item 誠実な成員の署名は偽造できず、署名されたメッセージの内容が変更されても、それを検知することができる。
%   \item 誰でも成員の署名の信憑性を検証することができる。
% \end{enumerate}

\section{問題解決の要件}
先の問題が解決されるためには、成員達が合意された約束を遵守する状態に至ることを示す他に、
下記の3つの要件が満たされている必要がある。

\subsection{履歴を定義}
第1に、社会契約の成立のために必要な履歴とはいかなるものかを定義する必要がある。
これは社会指標を計算するために各成員が記録すべき過去のある集団の何らかの状態である。

\subsection{社会指標の決定方法}
第2に、履歴から社会指標を計算する方法を定義する必要がある。
合意された約束が遵守されるようにするため、
この社会指標は約束を履行する成員の社会指標が上がり、
約束を反故にする成員の社会指標が下がるように設計する必要がある。

\subsection{成員の振る舞い}
第3に、各成員の振る舞いによってのみ、履歴が共有されて各成員の社会指標を計算でき、その振る舞いを記述可能である必要がある。
これは外部の強制執行力が存在しない場合、あらゆる履歴も計算された社会指標も、
その正当性を保証した状態で外部に記録共有することができないためである。