\chapter{問題提起}
\section{本論の問題提起}
これまでのゲーム理論を用いた社会契約の研究では、
どのようなプロセスで道徳的な規範が生じ維持され、社会契約が成立するかについて議論されていたが、
成員のどのような振る舞いによって、そのプロセスが進行しているのか記述することができなかった。
その最もたる障害は強制執行力の存在である。
先にも述べたとおり、強制執行力が集団の外部に存在する場合、それは別の社会契約の成立を意味する。
それ故、こうした仮定の上で社会契約の説明を試みても問題が堂々巡りに陥ってしまい、完全な社会契約のモデルを構築には至れない。
一方、外部の強制執行力が存在しない場合については、
社会の過去の歴史が決定する社会指標を利用することで社会契約が成立するとされている(Binmore 2005\cite{binmore2005})が、
この研究は還元主義的なアプローチをとっており、
社会契約に必要な歴史の定義や、社会指標の計算方法、成員達の具体的な振る舞いを記述することは困難である。

本研究では、改めて外部の強制執行力が存在しない場合について、
「合意された約束の履行を成員達に強制する力」を生じさせることが可能なのかという問いに取り組む。
その過程で、どのように歴史を決定し、そこからどのように社会指標を導き出し、
成員達がそれに基づいてどのように振る舞えば社会契約が成立するのかを計算可能なレベルで明確にする。

\section{問題解決の要件}
先の問題が解決されるためには、成員達が合意された約束を遵守する状態を作り出せていることを示す他に、
下記の3つの要件が満たされている必要がある。

\subsection{歴史を定義する}
第1に、社会契約の成立のために必要な歴史とはいかなるものかを定義する必要がある。
これは社会指標を計算するために各成員が記録すべき過去のある集団の何らかの状態である。

\subsection{社会指標の決定方法を定義する}
第2に、歴史から社会指標を計算する方法を定義する必要がある。
合意された約束が遵守されるようにするため、
この社会指標は約束を履行する成員の社会指標が上がり、
約束を保護にする成員の社会指標が下がるように設計する必要がある。


\subsection{成員の振る舞いとして記述できる}
第3に、各成員の振る舞いによってのみ、歴史が共有されて各成員の社会指標を計算でき、その振る舞いを記述可能である必要がある。
これは外部の強制執行力が存在しない場合、あらゆる歴史も計算された社会指標も、
その正当性を保証した状態で外部に記録共有することができないためである。