\chapter{問題提起}
\section{本論の問題提起}
これまでのゲーム理論を用いて社会契約のメカニズムを説明しようとする研究では、
強制執行力の存在が仮定されている、もしくは外部の強制執行力の存在が仮定されている上で議論が進められていた。
外部の強制執行力が存在している場合については、
別の社会契約が成立している必要があるため、その別の社会契約が成立しているメカニズムを説明する必要が生じ問題が堂々巡りに陥る。
一方、外部の強制執行力が存在しない場合については、
社会の過去の歴史が決定する社会指標を利用することで社会契約が成立するとされている(Binmore 2005\cite{binmore2005})が、
この研究は還元主義的であり具体的に成員達がどのような振る舞いをすることで、そうした歴史が決定されどのように社会指標を導き出すのかを決定することは困難であった。

そこで本研究では、改めて外部の強制執行力が存在しない場合について、合意された約束を成員達に遵守させることが可能なのかという問題に取り組み、
成員達のどのような振る舞いによって、歴史を決定し、社会指標を導き出せばよいのかを計算可能なレベルで明確にする。


\section{問題解決の要件}
先の問題が解決されるためには、成員達が合意された約束を遵守する状態を作り出せていることを示す他に、
下記の3つの要件が満たされている必要がある。

\subsection{歴史を定義する}
第1に、社会契約の成立のために必要な歴史とはいかなるものかを定義する必要がある。
これは社会指標を計算するために各成員が記録すべき過去のある集団の何らかの状態である。


\subsection{社会指標の計算方法を定義する}
第2に、歴史から社会指標を計算する方法を定義する必要がある。
合意された約束が遵守されるようにするため、
この社会指標は約束を履行する成員の社会指標が上がり、
約束を保護にする成員の社会指標が下がるように設計する必要がある。


\subsection{成員の振る舞いのみで記述できる}
第3に、各成員の振る舞いによってのみ、歴史が共有されて各成員の社会指標を計算でき、その振る舞いを記述可能である必要がある。
これは外部の強制執行力が存在しない場合、あらゆる歴史も計算された社会指標も、
その正当性を保証した状態で外部に記録共有することができないためである。