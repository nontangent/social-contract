\section{行動観察不可の条件}
   仮にこのようなシステムが存在していたとして,このシステムからインセンティブ設計を行う際には,
  商取引の当事者である$seller$と$buyer$の行動を観察することができないという条件(以後,「行動観察不可の条件」と呼ぶ.)をおくべきである.
  なぜなら,商取引が失敗した場合に$seller$と$buyer$のどちらに非があるかを確かめるには,
  第3者に依存せずに動作の正当性が保証されているという前提に反して,誰かしらの第3者に依存する必要があるからである.

  例えば,りんごの売買契約を結んだ$seller$と$buyer$がいて,
  どちらかがシステムにその商取引の失敗を報告したとする.
  ここで$seller$と$buyer$のどちらが不正行為を行ったかを知るために,
  代理人を選出して調査を行うことや,
  りんごにシステムから不正行為を検知するためのセンサーをあらかじめ埋め込むなどの方法がある.
  しかし,どちらも代理人やセンサーの正当性を保証する製造者などの第3者に依存してしまう.
  また,代理人や製造者が正当性が保証されたシステムの一部であると仮定しても,
  第3者である彼らの正当性が保証されているためには,
  別の「第3者に依存しない仲介システム」を用いて彼らと商取引を結んでいなければならず,
  その仮定は別の「第3者に依存しない仲介システム」という第3者に依存していることを前提としてしまう.

  つまり,商取引において不正行為を防止するためには,「行動観察不可の条件」を満たす「第3者に依存しない仲介システム」が存在していなければならないのである.
