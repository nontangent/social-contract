\section{不正防止の不可能性の証明}
\subsection{前提の整理}
  \begin{itemize}
    \item システムは$buyer$によって報告された商取引の結果を観察できる.
    % \vspace{-12mm}
    \item システムは$seller$と$buyer$がどの戦略を選んだかはわからない.
    % \vspace{-12mm}
    \item 商取引に参加する$player$は合理的に(利得の期待値が最も高い)戦略を決定する.
    % \vspace{-12mm}
    \item システム内には追跡可能な通貨が存在しており,商取引にはその通貨が用いられる.
    % \vspace{-12mm}
    \item システムからは$player$の通貨の保有量を操作することができる.
    % \vspace{-12mm}
  \end{itemize}

\subsection{不正行為が起きない戦略組とその条件}
 表より,商取引で不正行為が起きないためには,$seller$と$buyer$のとる戦略組が$ (s^{seller}_1, s^{buyer}_1)$もしくは$(s^{seller}_1, s^{buyer}_2)$のいずれかに帰着しなければならない.

  ここで$player$が戦略$s^{player}_n$をとったときの利得$R$の期待値を$E(R|s^{player}_n)$とする.

  $(s^{seller}_1, s^{buyer}_1)$に帰着するためには,\\

  $E(R|s^{seller}_1)>E(R|s^{seller}_2)$
  かつ
  $E(R|s^{buyer}_1)>E(R|s^{buyer}_2)$
  かつ
  $E(R|s^{buyer}_1)>E(R|s^{buyer}_3)$
  かつ
  $E(R|s^{buyer}_1)>E(R|s^{buyer}_4)$ …条件①\\

  を満たす必要があり,$(s^{seller}_1, s^{buyer}_2)$に帰着する場合は,\\

  $E(R|s^{seller}_1)>E(R|s^{seller}_2)$
  かつ
  $E(R|s^{buyer}_2)>E(R|s^{buyer}_1)$
  かつ
  $E(R|s^{buyer}_2) > E(R|s^{buyer}_3)$
  かつ
  $E(R|s^{buyer}_2) > E(R|s^{buyer}_4)$ …条件②\\

  を満たす必要がある.\\

   つまり,不正行為を防ぐ商取引のインセンティブ設計を行うためには,条件①もしくは条件②を満たす$(r^{seller}_{success}, r^{seller}_{failure}, r^{buyer}_{success}, r^{buyer}_{failure})$の組を「商取引システム」から決定することができなければならない.
  そこで,不正行為を防ぐ商取引のインセンティブ設計が不可能であることを示すために,次の命題を証明する.

\subsection{命題}
条件①もしくは条件②のいづれかの条件を満たす$(r^{seller}_{success}, r^{seller}_{failure}, r^{buyer}_{success}, r^{buyer}_{failure})$の組を商取引システムから決定することはできない.

\subsection{証明}
$player$が戦略$s^{player}_n$をとる確率を$p^{player}_{n}$と表す.$(0 \leq p^{player}_{n} \leq 1)$\\

$seller$について,各戦略の期待利得は以下のように表せる. \\
$E(R|s^{seller}_1)=p^{buyer}_1{r}^{seller}_{success}+p^{buyer}_2r^{seller}_{suceess}+p^{buyer}_3r^{seller}_{failure}+p^{buyer}_4r^{seller}_{failure}$ \\

$ E(R |s^{seller}_2) = p^{buyer}_1 (goods + r^{seller}_{failure}) + p^{buyer}_2 (goods + r^{seller}_{success}) + p^{buyer}_3 (goods + r^{seller}_{failure} ) + p^{buyer}_4 (goods + r^{seller}_{success})$ \\

$ = goods + p^{buyer}_1 r^{seller}_{failure} + p^{buyer}_2 r^{seller}_{success} + p^{buyer}_3 r^{seller}_{failure} + p^{buyer}_4 r^{seller}_{success}$ \\

ここで,$E(R |s^{seller}_1) > E(R |s^{seller}_2)$ を満たすためには, \\
$p^{buyer}_1 {r}^{seller}_{success} + p^{buyer}_2 r^{seller}_{suceess} + p^{buyer}_3 r^{seller}_{failure} + p^{buyer}_4 r^{seller}_{failure}$ \\

$ > goods + p^{buyer}_1 r^{seller}_{failure} + p^{buyer}_2 r^{seller}_{success} + p^{buyer}_3 r^{seller}_{failure} + p^{buyer}_4 r^{seller}_{success}$ \\

 $\therefore p^{buyer}_1 {r}^{seller}_{success} +  p^{buyer}_4 r^{seller}_{failure} > goods + p^{buyer}_1 r^{seller}_{failure} + p^{buyer}_4 r^{seller}_{success}$ \\

$\therefore p^{buyer}_1 {r}^{seller}_{success} +  p^{buyer}_4 r^{seller}_{failure} - goods + p^{buyer}_1 r^{seller}_{failure} - p^{buyer}_4 r^{seller}_{success} > 0$ \\

$\therefore p^{buyer}_1(r^{seller}_{success} - r^{seller}_{failure}) - p^{buyer}_4(r^{seller}_{success} - r^{seller}_{failure} ) - goods > 0$ \\

$\therefore (p^{buyer}_1 - p^{buyer}_4)(r^{seller}_{success} - r^{seller}_{failure}) - goods> 0$ \\

$\therefore (p^{buyer}_1 - p^{buyer}_4)(r^{seller}_{success} - r^{seller}_{failure} - \frac{ goods }{p^{buyer}_1 - p^{buyer}_4})> 0$ \\
を満たす必要がある.つまり,\\

$ p^{buyer}_1>p^{buyer}_4$のときは, \\

$ r^{seller}_{success} - r^{seller}_{failure} > \frac{ goods }{p^{buyer}_1 - p^{buyer}_4}$\\

$ p^{buyer}_1 < p^{buyer}_4$のときは,\\

$ r^{seller}_{success} - r^{seller}_{failure} < \frac{ goods }{p^{buyer}_1 - p^{buyer}_4}$\\

を満たせば,$E(R |s^{seller}_1) > E(R |s^{seller}_2)$である.
なお,$ p^{buyer}_1 = p^{buyer}_4$のときは,\\

$E(R |s^{seller}_1) > E(R |s^{seller}_2)$は成り立たない.\\

また,$ {buyer}$について,各戦略の期待利得は以下のように表せる.\\

$ E(R|s^{buyer}_1) = p^{seller}_1 (goods + r^{buyer}_{success}) + p^{seller}_2 r^{buyer}_{failure}$ \\

$ E(R|s^{buyer}_2)=p^{seller}_1 (goods + r^{buyer}_{success}) + p^{seller}_2 r^{buyer}_{success}$ \\

$ E(R|s^{buyer}_3)=p^{seller}_1 (goods + r^{buyer}_{failure}) + p^{seller}_2 r^{buyer}_{failure}$ \\

$ E(R|s^{buyer}_4)=p^{seller}_1 (goods + r^{buyer}_{failure}) + p^{seller}_2 r^{buyer}_{success}$ \\

\subsubsection{条件①が成り立たないことの証明}

ここで,条件①の必要条件である
$E(R|s^{buyer}_1)>E(R|s^{buyer}_2)$
かつ
$E(R|s^{buyer}_1)>E(R|s^{buyer}_3)$
かつ
$E(R|s^{buyer}_1)>E(R|s^{buyer}_4)$
を満たすためには,\\

$ E(R|s^{buyer}_1) > E(R|s^{buyer}_2)$ \\

$\therefore p^{seller}_1 (goods + r^{buyer}_{success}) + p^{seller}_2 r^{buyer}_{failure} > p^{seller}_1 (goods + r^{buyer}_{success}) + p^{seller}_2 r^{buyer}_{success}$ \\

$\therefore p^{seller}_2 r^{buyer}_{failure} - p^{seller}_2 r^{buyer}_{success} > 0$ \\

$\therefore p^{seller}_2 (r^{buyer}_{failure} - r^{buyer}_{success}) > 0$ \\

つまり,$ p^{seller}_2 > 0$ かつ $ r^{buyer}_{failure} - r^{buyer}_{success} > 0$

\begin{equation}
  p^{seller}_2 > 0
\end{equation}

\begin{equation}
\label{quad1}
  0 > r^{buyer}_{success} - r^{buyer}_{failure}
\end{equation}

$ E(R|s^{buyer}_1) > E(R|s^{buyer}_3)$ \\

$ \therefore p^{seller}_1(goods + r^{buyer}_{success}) + p^{seller}_2r^{buyer}_{failure}>
  p^{seller}_1(goods + r^{buyer}_{failure}) + p^{seller}_2r^{buyer}_{failure}$\\

$ \therefore p^{seller}_1 (r^{bueyer}_{success} - r^{buyer}_{failure}) > 0$\\

つまり,$ p^{seller}_1 > 0$ かつ $r^{bueyer}_{success} - r^{buyer}_{failure} > 0$\\

\begin{equation}
   p^{seller}_1 > 0
\end{equation}
\begin{equation}
   \label{quad2}
   r^{buyer}_{success} - r^{buyer}_{failure} > 0
\end{equation}

$ E(R|s^{buyer}_1) > E(R|s^{buyer}_4)$ \\

$\therefore p^{seller}_1 (goods + r^{buyer}_{success}) + p^{seller}_2 r^{buyer}_{failure} > p^{seller}_1 (goods + r^{buyer}_{failure}) + p^{seller}_2 r^{buyer}_{success}$ \\

$\therefore p^{seller}_1(r^{buyer}_{success} - r^{buyer}_{failure}) + p^{seller}_2(r^{buyer}_{failure} - r^{buyer}_{success}) > 0$ \\

$ (p^{seller}_1 - p^{seller}_2)(r^{buyer}_{success} - r^{buyer}_{failure}) > 0$\\

\begin{equation}
\label{quad3}
  (p^{seller}_1 - p^{seller}_2)(r^{buyer}_{success} - r^{buyer}_{failure}) > 0
\end{equation}

の3つを満たす必要がある.しかし,(\ref{quad1})と(\ref{quad2})は矛盾するため,
$(r^{seller}_{success}, r^{seller}_{failure}, $
$r^{buyer}_{success}, r^{buyer}_{failure})$ \\
がいかなる実数の組でも条件①は成り立たない. \\


\subsubsection{条件②が成り立たないことの証明}
また,条件②の必要条件である
$E(R|s^{seller}_1)>E(R|s^{seller}_2)$
かつ
$E(R|s^{buyer}_2)>E(R|s^{buyer}_1)$
かつ
$E(R|s^{buyer}_2) > E(R|s^{buyer}_3)$
かつ
$E(R|s^{buyer}_2) > E(R|s^{buyer}_4)$
を満たすためには,

$ E(R|s^{buyer}_2) > E(R|s^{buyer}_1)$ \\

$\therefore p^{seller}_1 (goods + r^{buyer}_{success}) + p^{seller}_2 r^{buyer}_{success} > p^{seller}_1 (goods + r^{buyer}_{success}) + p^{seller}_2 r^{buyer}_{failure}$\\

$\therefore p^{seller}_2 (r^{buyer}_{success} - r^{buyer}_{failure}) > 0$\\

$0 \leq p^{seller}_2 \leq 1$より,\\

$p^{seller}_2 > 0$かつ
$r^{buyer}_{success} - r^{buyer}_{failure} > 0$\\

\begin{equation}
  p^{seller}_2 > 0
\end{equation}

\begin{equation}
  \label{quad2-1}
  r^{buyer}_{success} - r^{buyer}_{failure} > 0
\end{equation}

$ E(R|s^{buyer}_2) > E(R|s^{buyer}_3)$\\

$\therefore p^{seller}_1 (goods + r^{buyer}_{success}) + p^{seller}_2 r^{buyer}_{success} > p^{seller}_1 (goods + r^{buyer}_{failure}) + p^{seller}_2 r^{buyer}_{failure}$\\

$\therefore p^{seller}_1 (r^{buyer}_{success} - r^{buyer}_{failure}) + p^{seller}_2 (r^{buyer}_{success} - r^{buyer}_{failure}) > 0$\\

$\therefore (p^{seller}_1 + p^{seller}_2)(r^{buyer}_{success} - r^{buyer}_{failure}) > 0$\\

$ p^{seller}_1 + p^{seller}_2 = 1 > 0$より,\\

$ r^{buyer}_{success} - r^{buyer}_{failure} > 0$\\

\begin{equation}
  \label{quad2-2}
  r^{buyer}_{success} - r^{buyer}_{failure} > 0
\end{equation}

$ E(R|s^{buyer}_2) > E(R|s^{buyer}_4)$\\

$\therefore p^{seller}_1 (goods + r^{buyer}_{success}) + p^{seller}_2 r^{buyer}_{success} > p^{seller}_1 (goods + r^{buyer}_{failure}) + p^{seller}_2 r^{buyer}_{success}$\\

$\therefore p^{seller}_1(r^{buyer}_{success} - r^{buyer}_{failure}) > 0$\\

$ p^{seller}_1 > 0$かつ
$ r^{buyer}_{success} - r^{buyer}_{failure} > 0$\\

\begin{equation}
  p^{seller}_1 > 0
\end{equation}

\begin{equation}
  \label{quad2-3}
  r^{buyer}_{success} - r^{buyer}_{failure} > 0
\end{equation}

ここで,(\ref{quad2-1}),(\ref{quad2-2}),(\ref{quad2-3})を満たす
$ (r^{buyer}_{success}, r^{buyer}_{failure})$の組はシステムから決定できるため,\\

$p^{seller}_2>0$かつ$p^{seller}_1>0$であれば,

$E(R|s^{buyer}_2)>E(R|s^{buyer}_1)$
かつ
$E(R|s^{buyer}_2) > E(R|s^{buyer}_3)$
かつ
$E(R|s^{buyer}_2) > E(R|s^{buyer}_4)$
を満たすことができる.\\

ここで仮に$p^{seller}_2>0$かつ$p^{seller}_1>0$が成り立ち$ buyer$が戦略$ s^{buyer}_2$を選択するととする.\\

このとき,$buyer$が各戦略をとる確率$(p^{buyer}_1, p^{buyer}_2, p^{buyer}_3, p^{buyer}_4)$は$ (0, 1, 0, 0)$と表せる.\\

ここで,$ p^{buyer}_1 = p^{buyer}_4 = 0$のため,
$E(R |s^{seller}_1) > E(R |s^{seller}_2)$は成り立たない.\\

それゆえに,条件②は成り立たない.\\

以上より,条件①もしくは条件②のいづれかの条件を満たす
$(r^{seller}_{success}, r^{seller}_{failure},$
$r^{buyer}_{success},r^{buyer}_{failure})$の組を「商取引システム」から決定することはできない.\\

