\chapter*{謝辞}
本研究を進めるにあたり、2016年度秋学期から所属していた研究軍団NECO及び
RG(村井・徳田・中村・楠本・高汐・バンミーター・植原・三次・中澤・手塚・武田合同研究会)の皆様、
慶應SFCの教員の皆様、大学進学を支援してくださった家族に感謝します。
中でも早稲田大学 大学院経営管理研究科 教授 斉藤賢爾氏には大変お世話になりました。
本研究が全く問題意識をはっきりとしない頃から議論にお付き合いいただいたことは勿論、
斉藤氏が著書「信用の新世紀 ブロックチェーン後の未来」\cite{saito2017}の中で描かれるこの社会の未来像は
この研究の大きなモチベーションになりました。
彼のデジタルマネーに関する思想に出会わなければ、
本研究における問題意識を抱くこともなく社会契約を研究対象として選ぶこともなかったと言えます。
また、代々KGLとしてNECOの運営を担ってくださった阿部涼介氏、菅藤 佑太氏、島津 翔太氏、渡邉 聡紀氏にも感謝いたします。
特に慶應義塾大学 政策・メディア研究科 博士課程の阿部涼介氏は自分にも他人にも厳しい方でしたが、
加えて、彼の研究への姿勢から多くを学ぶことができました。
日頃より、RGの運営や講義、研究発表でお世話になった
慶應義塾大学教授 村井純博士、同学部教授 中村修博士、同学部教授 楠本博之博士、同学部教授 高汐一紀博士、
同学部教授 Rodney D.Van Meter III 博士、同学部准教授 植原啓介博士、
同学部教授 三次仁博士、同学部教授 中澤仁博士、同学部教授 武田圭史博士、
同大学政策・メディア研究 科特任准教授 佐藤雅明博士、同大学政策・メディア研究科特任教授 鈴木茂哉博士もありがとうございました。
また、RGの運営に携わってくれたRG Coordinatorの皆様にも感謝します。
最後に、慶應SFCという素敵なキャンパスでの大学生活を支えてくださった
教員の皆さまや友人、家族に感謝です。
