~ \\
% 背景(分野外の研究者でも理解できる内容)
  インターネットのグローバル化やBitcoin\cite{nakamoto2008bitcoin}をはじめとするブロックチェーン技術の登場により、
  インターネット上のサービスの正当性を特定の国家の法によって保証することは困難になりつつある。
  我々は社会契約の概念をインターネットに適用することがこの問題の解決に繋がると考え、その抽象化を試みている。
% 背景(関連分野の)
  社会契約とは、ある集団においてその成員達が合意された約束を必ず履行する状態に至るプロセスである。
  現代においては、この概念は交渉理論や進化ゲーム理論を用いて、
  正義や倫理、道徳といった概念とともに研究されている\cite{sep-game-ethics}が、
  その多くは利得の分配を強制執行する力の存在が暗黙的に仮定されている。
  この強制執行力はそれ自身が利得を分配する約束であるため、
  集団の外部にその存在を仮定することは別の社会契約の成立を仮定することに等しい。
  それゆえ、社会契約全体をモデリングするためには、強制執行力が集団の内部で生じるメカニズムを解明する必要がある。
  Binmoreの2005年の研究\cite{binmore2005}はこれを試みているが、
  このアプローチは還元主義的なため強制執行力を生じさせる成員の振る舞いが具体的にどういったものか明確でない。
% 問題(1文)
  本研究では、強制執行力が集団の外部に存在しない場合に社会契約が成立するのかという問いに取り組む。
% 論文の趣旨
  ここで我々は、強制執行力を自己組織化させる成員の振る舞いがビザンチン将軍問題の署名付きの解法\cite{lamport1982}によって説明でき、
  その振る舞うに従う成員が集団の一部のみであっても社会契約が成立することを示す。
% 結果(先行研究との比較を含める)
  そのために我々はゲーム理論とビザンチン将軍問題の解法を用いて複雑系として社会契約をモデリングし、
  ランダムな8体のエージェント(8種類・重複あり)の相互作用の果てに社会契約が成立するか確かめるシミュレーションを行った。
  64000回の施行の結果、誠実に振る舞うエージェントが過半数を超える場合、サンプリングした全てのケースで社会契約は成功した。
% まとめ
  これにより、社会契約の成立に必要となる歴史の定義や社会指標の決定方法、各成員の振る舞いを計算可能なレベルで明確にし、
  同時に、強制執行力が集団の外部に存在しない場合でも過半数の成員が誠実であれば社会契約が必ず成立することを示した。
% 今後について
  この誠実な成員の数と社会契約の成立の関係性はヒューリスティックな解であるため、
  より厳密な解を得るには大規模な総当り実験を行うか理論的な解明がなされる必要がある。
  仮に$3f+1$よりゆるい条件で合意された約束が必ず履行される状態に到れるならば、
  時間経過とともに$f$が増加する場合に$3f+1$以下の人数でフォールトトレラントを達成できる実用的なBFTアルゴリズムを示せる可能性がある。
  また、現実の社会を考えたとき、集団の人数は時間経過とともに変化するものであるため、それを考慮して社会契約を更新するモデルの構築が必要である。
  本研究がそういった新たな研究の足がかりになることを願う。
~ \\