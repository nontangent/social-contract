% 分散型の評判システムと計算可能な複雑系としての社会契約
% Distributed Reputation System and Social Contract as a Computable Complex System
% 非中央集権型の評判システムと計算可能な複雑系としての社会契約
% Decentralized Reputation System and Social Contract as a Computable Complex System

~ \\
% 背景
% 社会契約とは
  社会契約とは、ある集団において、「各成員に合意された約束の履行を強制する力」を生じさせるプロセスである。
% 先行研究
  この力がどのようにして生じるのかについては、交渉理論や進化ゲーム理論を用いて研究がなされきた(Verbeek Bruno Christopher 2020)\cite{sep-game-ethics}が、
  ほとんどの研究は、そのゲームにおける利得の分配を強制執行する力の存在が暗黙的に仮定されている。
  この強制執行力が集団の外部に存在することを仮定する場合、それは別の社会契約が成立していることを意味するため、社会契約を完全に説明することはできない。
  逆に、外部に強制執行力の存在を仮定しない場合、過去の歴史とそこから決定される社会指標を用いることで社会契約は成立する(Binmore 2005)。
% 問題
  しかしながら、その研究では内部的に強制執行力を生じさせるメカニズムについては還元主義的にしか説明されておらず、
  その強制執行力を生じさせる成員の振る舞いを明確に記述することはできなかった。
% 提案手法
  本研究では、
  まず、外部の強制執行力として、「報告された約束の記録(歴史)から各成員の評判(社会指標)を決定するシステム」の存在を仮定し、
  そのシステムが各成員に合意された約束の履行を強制する条件について考えた。
  次に、そのシステムを、ビザンチン将軍問題の解法を用い、各成員の振る舞いによって自己組織化された分散システムとして設計した。
% 実験結果
  この分散システムを、マルチエージェントシミュレーションを用いて検証した結果、
  成員の構成によっては、一部の成員が想定されない振る舞いを行った場合であっても、
  合意された約束が必ず履行される状態に至ることがわかった。
% 結論
  これにより、外部の強制執行力が存在しない場合でも、成員の構成次第で社会契約が成立することを示せ、
  社会契約の成立に必要となる歴史の定義や社会指標の計算方法、各成員の振る舞いを計算可能なレベルで明確することができた。
% 貢献
  もし将来、コンピューターを用いた人々の社会契約が必要とされる時が来るならば、
  今回の研究がその実現に何らかの形で貢献することを願う。
~ \\