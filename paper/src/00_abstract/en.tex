~ \\
% 背景
% 社会契約とは
  Social contract is a process that gives rise to "the power to compel each member to fulfill an agreed promise" in a group.
% 先行研究
  % この力がどのようにして生じるのかについては、交渉理論や進化ゲーム理論を用いて研究されきた(Verbeek Bruno Christopher 2020)\cite{sep-game-ethics}が、
  % 大半の研究は、そのゲームにおける利得の分配を強制執行する力の存在が暗黙的に仮定されている。
  How this power arises has been studied using bargaining theory and evolutionary game theory (Verbeek Bruno Christopher 2020)\cite{sep-game-ethics}, 
  but Most studies implicitly assume the existence of an enforcement the distribution of gains in the game.

  % この強制執行力が集団の外部に存在することを仮定する場合、それは別の社会契約が成立していることを意味するため、
  % 社会契約を完全に説明することはできない。
  If we assume that this enforcement exists outside the group, 
  it means that another social contract is in place, and thus we cannot fully explain the social contract.
  % 逆に、外部に強制執行力の存在を仮定しない場合、過去の歴史とそこから決定される社会指標を用いることで社会契約は成立する(Binmore 2005)。
  On the contrary, when the existence of external enforcement is not assumed, 
  the social contract is established by using past history and social indicators determined from it (Binmore 2005).
% 問題
  % しかしながら、その研究では内部的に強制執行力を生じさせるメカニズムについては還元主義的にしか説明されておらず、
  % その強制執行力を生じさせる成員の振る舞いを明確に記述することはできなかった。
  However, that study only explained the mechanism of internal enforcement in a reductionist way, 
  and failed to clearly describe the behavior of the members that generate the enforcement.
% 提案手法
  % 本研究では、
  % まず、外部の強制執行力として、「各成員から報告された約束の結果(歴史)を記録し、それに基づいて各成員の評判(社会指標)を決定するシステム」の存在を仮定し、
  % そのシステムが各成員に合意された約束の履行を強制する条件について考えた。
  % 次に、そのシステムを、ビザンチン将軍問題の署名付きの解法を用い、各成員の振る舞いによって自己組織化された分散システムとして設計した。
  In this study. 
  First, we assume the existence of "a system that records the results of promises reported by each member (history) 
  and determines each member's reputation (social index) based on these records" as an external enforcement force. 
  We considered the conditions under which the system would force each member to fulfill the agreed-upon promise.
% 実験結果
  % この分散システムを、マルチエージェントシミュレーションを用いて検証した結果、
  % 成員の構成によっては、一部の成員が想定されない振る舞いを行った場合でも、最終的に合意された約束が必ず履行される状況を作り出せることがわかった。
  We have verified this distributed system using multi-agent simulation. 
  We found that, depending on the composition of the members, 
  it is possible to create a situation in which the agreed-upon promise is always fulfilled, 
  even if some members behave in unexpected ways.
% 結論
  % これにより、外部の強制執行力が存在しない場合でも、成員の構成次第で社会契約が成立することを示せ、
  % 社会契約の成立に必要となる歴史の定義や社会指標の計算方法、各成員の振る舞いを計算可能なレベルで明確に定義することができた。
  This allows us to show that even in the absence of external enforcement, 
  a social contract can be established depending on the composition of the members.
  The definition of history, the method of calculating social indicators, and the behavior of each member, 
  which are necessary for the establishment of a social contract, can be clearly defined at a computable level.
% 貢献
  % もし将来、コンピューターを用いた人々の社会契約が必要とされる時が来るならば、
  % 今回の研究がその実現に何らかの形で貢献することを願う。
  If there is ever a time in the future when a computer-based social contract between people is needed. 
I hope that this research will contribute in some way to its realization.
~ \\

