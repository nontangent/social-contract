~ \\
% 背景(分野外の研究者でも理解できる内容)
% インターネットのグローバル化やBitcoin\cite{nakamoto2008bitcoin}を始めとするブロックチェーン技術の登場により、
% インターネット上のサービスの正当性を特定の国家の法によって保証することは困難になりつつある。
% 我々は、社会契約の概念をインターネットに適用することがこの問題の解決に繋がると考え、
% それを計算可能なレベルまで抽象化することを試みている。
With the globalization of the Internet and the emergence of blockchain technologies such as Bitcoin\cite{nakamoto2008bitcoin},
it is becoming increasingly difficult to guarantee the legitimacy of services on the Internet through the laws of a particular nation.
We believe that applying the concept of social contract to the Internet will help solve this problem, 
and we are attempting to abstract it to a computable level.
% 背景(関連分野の)
% 社会契約とは、ある集団においてその成員達が合意された約束を必ず履行する状態に至るプロセスである。
% 現代においては、この概念は交渉理論や進化ゲーム理論を用い、
% 正義や倫理、道徳といった概念とともに研究がなされている\cite{sep-game-ethics}が、
% その多くは利得の分配を強制執行する力の存在が暗黙的に仮定されている。
The social contract is the process by which a group of people reach a state in which they always fulfill their agreed-upon promises. 
In modern times, this concept has been studied using bargaining theory and evolutionary game theory, 
along with concepts such as justice, ethics, and morality\cite{sep-game-ethics}. 
Many of them implicitly assume the existence of the \emph{enforcement} to the distribution of payoff.
% この強制執行力は、それ自身が利得を分配する約束であるため、
% 集団の外部にその存在を仮定することは別の社会契約の成立を仮定することに等しい。
% それゆえ、社会契約全体をモデリングするためには、強制執行力が集団の内部で生じるメカニズムを解明する必要がある。
% Binmoreの2005年の研究\cite{binmore2005}はこれを試みているが、
% このアプローチは還元主義的なため強制執行力を生じさせる成員の振る舞いが具体的にどういったものか明確でない。
Since this \emph{enforcement} is itself a promise to distribute payoff, 
assuming its existence outside the group is equivalent to assuming the formation of another social contract.
Therefore, in order to model the social contract as a whole, 
it is necessary to elucidate the mechanisms by which \emph{enforcement} arises within a group.
Binmore's 2005\cite{binmore2005} study attempts to solve this problem, 
but because this approach is reductionist, 
it is not clear what exactly are the behaviors of members that give rise to \emph{enforcement}.
% 問題(1文)
% 本研究では、強制執行力が集団の外部に存在しない場合に社会契約が成立するのかという問いに取り組む。
In this study, we research the question of whether a social contract can be established 
when the \emph{enforcement} does not exist outside the group.
% 論文の趣旨
% ここで我々は、強制執行力を自己組織化させる成員の振る舞いがビザンチン将軍問題の署名付きの解法\cite{lamport1982}によって説明でき、
% 集団の一部のみがそのように振る舞う場合であっても社会契約が成立することを示す。
Here we show that the behavior of members who self-organize \emph{enforcement} can be explained 
by the signed solution to the Byzantine Generals Problem\cite{lamport1982}, 
and that a social contract can be established even when only a part of the group behaves in this way.
% 結果(先行研究との比較を含める)
% そのために我々はゲーム理論とビザンチン将軍問題の解法を用いて社会契約を成員の振る舞いによる複雑系としてモデリングし、
% ランダムな8体のエージェント(8種類・重複あり)の相互作用の果てに社会契約が成立するか確かめるシミュレーションを行った。
% 64000回の施行の結果、誠実に振る舞うエージェントが過半数を超える場合、サンプリングした全てのケースで社会契約は成功した。
For this purpose, we modeled the social contract as a complex system with members' behavior 
using game theory and the solution of the Byzantine General Problem. 
We conducted a simulation to see if a social contract can be established 
after the interaction of eight random agents (eight types, with overlaps).
As a result of 64,000 runs, the social contract was successful in all the cases sampled 
when the majority of agents behaved honestly.
% まとめ
% これにより、社会契約の成立に必要となる歴史の定義や社会指標の決定方法、各成員の振る舞いを計算可能なレベルで明確にし、
% 同時に、強制執行力が集団の外部に存在しない場合でも過半数の成員が誠実であれば社会契約が必ず成立することを示した。
In this way, we have been able to define the history required for the formation of a social contract, 
determine the social indicators, and clarify the behavior of each member on a computable level. At the same time, 
it showed that a social contract can always be established if the majority of members are sincere, 
even if there is no \emph{enforcement} outside the group.
% 今後について
% この誠実な成員と社会契約の成立の関係性はヒューリスティックな解であるため、
% より厳密な解を得るには大規模な総当り実験を行うか理論的な解明がなされる必要がある。
Since the relationship between sincere members and the formation of a social contract is a heuristic solution, 
a large-scale brute force experiment or theoretical clarification is needed to obtain a more rigorous solution.
% 仮に$3f+1$よりゆるい条件で合意された約束が必ず履行される状態に到れるならば、
% 時間経過とともに$f$が増加する場合に$3f+1$以下の人数でフォールトトレラントを達成できる実用的なBFTアルゴリズムを示せる可能性がある。
If we can reach a state in which promises agreed to under looser conditions than $3f+1$ are always fulfilled, 
then We may be able to show a practical BFT algorithm that can achieve fault-tolerance 
with less than $3f+1$ people when $f$ increases with time.
% また、現実の社会を考えたとき、集団の人数は時間経過とともに変化するものであるため、それを考慮して社会契約を更新するモデルの構築が必要である。
% 本研究がそういった新たな研究の足がかりになることを願う。
In addition, when we consider real society, the number of people in a group changes over time, and it is necessary to construct a model that takes this into account and updates the social contract.
I hope that this research will serve as a stepping stone for such new research.
~ \\

