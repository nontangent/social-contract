\subsection{不正が抑制される戦略組と期待利得の不等式}
「倫理ある商取引ゲーム」において、不正を防止するためには、
$seller$と$buyer$の戦略組を$(s^{seller}_1, s^{buyer}_1)$に帰着させる必要がある。.
そのためには$seller$が戦略$s^{seller}_1$をとった場合の期待利得$E(r|s^{seller}_1)$が
戦略$s^{seller}_2$をとった場合の期待利得$E(r|s^{seller}_2)$より大きく,
$buyer$が戦略$s^{buyer}_1$をとった場合の期待利得$E(r|s^{buyer}_1)$が
戦略$s^{buyer}_3$をとった場合の期待利得$E(r|s^{buyer}_3)$より大きくならなければならない.
つまりは,$E(r|s^{seller}_1) > E(r|s^{seller}_2)$かつ$E(r|s^{buyer}_1) > E(r|s^{buyer}_3)$を満たす
$(r^{seller}_{success}, r^{seller}_{failure}, r^{buyer}_{success}, r^{buyer}_{failure})$の組を
「商取引システム」から決定できる必要がある.

\subsection{$ seller $と$ buyer $の期待利得}

$ seller $と$ buyer $の各戦略の利得の期待値は以下のように表せる.\\


$ E(R|s^{seller}_1) = p^{buyer}_1 (r^{seller}_{success} + \epsilon^{seller}) + p^{buyer}_3 (r^{seller}_{failure} + \lambda^{seller}) $ \\

$ E(R|s^{seller}_2) = p^{buyer}_1 (goods + r^{seller}_{failure} + \lambda^{seller}) + p^{buyer}_3 (goods + r^{seller}_{failure} + \lambda^{seller}) $ \\

$ = goods + r^{seller}_{failure} + \lambda^{seller} $ \\

$ \because p^{buyer}_1 + p^{buyer}_3 = 1 $(「倫理ある商取引ゲーム」において、$ p^{buyer}_2 $と$ p^{buyer}_4 $は0であるため) \\

$ E(R|s^{buyer}_1) = p^{seller}_1 (goods + r^{buyer}_{success} + \epsilon^{buyer}) + p^{seller}_2 (r^{buyer}_{failure} + \lambda^{buyer}) $ \\

$ E(R|s^{buyer}_3) = p^{seller}_1(goods+r^{buyer}_{failure} + \lambda^{buyer}) + p^{seller}_2 (r^{buyer}_{failure} + \lambda^{buyer}) $ \\

$ = p^{seller}_1 goods + r^{buyer}_{failure} + \lambda^{buyer} $ \\

$ \because p^{seller}_1 + p^{seller}_2 = 1 $ \\


\subsection{$ seller $が誠実な戦略をとる条件}

$ E(R|s^{seller}_1) > E(R|s^{seller}_2) $ \\

$ \therefore p^{buyer}_1 (r^{seller}_{success} + \epsilon) + p^{buyer}_2 (r^{seller}_{failure} + \lambda) > goods + r^{seller}_{failure} + \lambda $ \\

$ \therefore p^{buyer}_1(r^{seller}_{success} + \epsilon) - p^{buyer}_1(r^{seller}_{failure} + \lambda) > goods $ \\

$ \therefore p^{buyer}_1(r^{seller}_{success} - r^{seller}_{failure} + \epsilon - \lambda) > goods $ \\

仮定より,$ \epsilon > \lambda $のため,
$ p^{buyer}_1 (r^{seller}_{success} - r^{seller}_{failure}) \geq goods $を満たせばよい.

$ 0 < p^{buyer}_1 $を仮定するならば,
$ r^{seller}_{success} - r^{seller}_{failure} \geq \frac{goods}{p^{buyer}_1} $


\subsection{$ buyer $が誠実な戦略をとる条件}

$ E(R|s^{buyer}_1) > E(R|s^{buyer}_3) $ \\

$ \therefore p^{seller}_1 (goods + r^{buyer}_{success}) + p^{seller}_2 r^{buyer}_{failure} > p^{seller}_1(goods+r^{buyer}_{failure}) + p^{seller}_2 r^{buyer}_{failure} $ \\

$ \therefore p^{seller}_1(r^{buyer}_{success} - r^{buyer}_{failure}) > 0 $ \\

$ 0 < p^{seller}_1 $を仮定するならば, \\

$ r^{buyer}_{success} - r^{buyer}_{failure} > 0 $ \\

上記をまとめると,$ 0<p^{buyer}_1 $かつ $ 0 < p^{seller}_{1} $を仮定した上で,\\

$ r^{seller}_{success} - r^{seller}_{failure} \geq \frac{goods}{p^{buyer}_1} $かつ$ r^{buyer}_{success} - r^{buyer}_{failure} > 0 $ \\

を満たせば,「倫理ある商取引ゲーム」で不正を防止することができる.


\subsection{信頼度 $ p^{player} $}

ここで,任意の$ player $が誠実な戦略($ p^{seller}_1, p^{buyer}_1 $のいづれか)をとる主観確率を$ p^{player}_1 $とすると,
$ p^{player}_1 $は各プレイヤーに対して誠実な戦略をとった割合$ HonestyStrategyRate(player, opportunity) $と任意の重み$ w^{player} $を用いて次のように表せる. \\

\begin{equation}
  p^{player}_1 \equiv \sum^{players}_{opp} w^{opp} HonestyStrategyRate(player, opp)
\end{equation}

\subsection{最低信頼度 $ T^{player} $}

しかし、「商取引システム」からは$ HonestyStrategyRate(player, opportunity) $は未知のため,
信頼度$ p^{player}_1 $を求めることができない.
そこで$ HonestyStrategyRate $の代わりに$ ReportedSuccessRate $を用い、
信頼度$ p^{player} $を計算するのと同じ重み$ w^{player} $の荷重総和をとったものを、
最低信頼度$ T^{player} $と定義する. \\

\begin{equation}
  T^{player} \equiv \sum^{players}_{opp} {w}^{opp} ReportedSuccessRate(player, opp)
\end{equation}

\subsection{最低信頼度を用いた条件}
ここで$ HonestyStrategyRate \geq ReportedSuccessRate $であるため,$ p^{player}_1 \geq T^{player} $がいえる.

ゆえに, \\

\begin{equation}
  r^{seller}_{success} - r^{seller}_{failure}  \geq \frac{goods}{T^{buyer}} \geq \frac{goods}{p^{buyer}_1}
\end{equation}

となる.

つまり、$ 0<p^{buyer}_1 $かつ $ 0 < p^{seller}_{1} $を仮定した上で,\\

\begin{equation}
  r^{seller}_{success} - r^{seller}_{failure} \geq \frac{goods}{T^{buyer}} かつ r^{buyer}_{success} - r^{buyer}_{failure} > 0
\end{equation}

を満たす$ (r^{seller}_{success}, r^{seller}_{failure}, r^{buyer}_{success}, r^{buyer}_{failure}) $の組を「商取引システム」から決定できれば、
「倫理ある商取引ゲーム」において不正を防止することができる。