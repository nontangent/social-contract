本節ではシミュレーションに用いる「評判システム」の仕様の詳細を紹介する。
完全な実装については、GitHubのソースコード\cite{mitamoto2021}を参照。

\subsection{約束の価値$C$}
約束が履行されることによる$promiser$が失う効用と$reporter$が得られる効用が等しいと仮定し、
それを約束の価値$C$とおき、$1$とした。

\begin{eqnarray}
  C &\equiv& 1 \\
    &=& c_{p2} - c_{p1} \\
    &=& c_{r1} - c_{r2}
\end{eqnarray}

\subsection{ReputationWeight}
最低信頼度$T(i, s)$を求めるため、$ReportedFulfillRate(i, j)$の荷重総和に用いる重み$w_k$を定義する.
この重み$w_k$は、任意の成員$i$が戦略$s_{p1}$をとる確率$p_{p1}$を推定する際に、
成員$i$と他の成員$j$における$ReportedFulfillRate(i, j)$をどの程度信頼するかを表している。
本実験では、成員$j$の「評判スコア」が全体に占める割合をReputationWeight$w_j$と定義する。
任意の成員$k$の「評判スコア」を$b_k$としたとき、$w_k$は次のように表せる。($n$は成員の人数)

\begin{equation*}
  w_k \equiv \frac{b_k}{\sum^{\{1,2,...,n\}}_{k}b_k}
\end{equation*}

\subsection{「履行」が報告された場合の「評判スコア」の変化}
$reporter$が「履行」を報告した場合、
$reporter$の「評判スコア」は約束の価値$C$だけ減り、$promisor$の「評判スコア」は$C$だけ増加するものとする。

\begin{gather}
  r_{ps} = C \\
  r_{rs} = -C \\
  r_{ps} + r_{rs} = 0
\end{gather}


\subsection{EscrowCost}
「失敗」が報告されたとき、$promisor$と$reporter$が失う「評判スコア」の合計を$EscrowCost$とする。
約束の価値$C$にエスクロー係数$E$を掛けたものとする.

\begin{eqnarray}
  EscrowCost  &\equiv& (r_{rs} - r_{rf}) + (r_{ps} - r_{pf}) \\
              &=& E \cdot C \label{escrowCost}
\end{eqnarray}

\subsection{EscrowCostの負担比率}
成員$i$を$promiser$、$j$を$reporter$とする。
「約束・評判ゲーム」を行うとき、$EscrowCost$の負担比率を両者の最低信頼度$T_i$を用いる.\\

\begin{equation}
  (r_{ps} - r_{pf}):(r_{rs} - r_{rf}) = T_j:T_i
\end{equation}

\subsection{EscrowCostの分配}
「失敗」が報告されたときに$EscrowCost$が消失すると、
「評判スコア」の総量が下がり、約束の価値$C$

総量を一定に保つために、$promisor$と$reporter$以外の全てのプレイヤーに、
$ReputationWeight$に応じて$EscrowCost$を分配する。
$promisor$と$reporter$を含まないのは、分配によるインセンティブ設計への影響をなくすためである。

\subsection{不正が防止される条件を満たす定義}
成員$j$を$promiser$、$k$を$reporter$とする。
このとき、\eqref{condition6-1}と\eqref{condition6-2}を満たす$r_{ps} - r_{pf}$と$r_{rs} - r_{rf}$を下記のように定義する。

\begin{eqnarray}
  r_{ps} - r_{pf} &\equiv& \frac{C}{T(i, s_{p1})} \label{condition6-3} \\
                  &\geq& \frac{c_{p2} - c_{p1}}{T(i, s_{p1})} \nonumber \\
  r_{rs} - r_{rf} &\equiv& \frac{C}{T(j, s_{r1})} \label{condition6-4} \\
                  &\geq& 0 \nonumber
\end{eqnarray}

\subsection{残高の変化量の組$ (r_{ps}, r_{pf}, r_{rs}, r_{rf}) $}
上記の定義からエスクロー係数$E$と残高の変化量の組$ (r_{ps}, r_{pf}, r_{rs}, r_{rf}) $は次のように求まる. \\
成員$i$を$promisor$、$j$を$reporter$とする。

\begin{eqnarray}
  E &=& - \frac{T(i, s_{p1}) + T(j, s_{r1})}{T(i, s_{p1}) \cdot T(j, s_{r1})} \\
  r_{ps} &=& 1 \\
  r_{pf} &=& 1 - \frac{1}{T(j, s_{r1})} \\
  r_{rs} &=& -1 \\
  r_{rf} &=& -1 - \frac{1}{T(i, s_{p1})}
\end{eqnarray}
