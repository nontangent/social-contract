\subsection{自己信頼}
ここで$ HonestStrategyRate(player, player) $と$ ReportedSuccessRate(player, player) $は1とする.これは$ player $が$ player $自身と行う商取引は必ず成功するためである.

\subsection{将来期待利得}
通常の「商取引ゲーム」においては誠実な戦略を取ってきた割合と報告された商取引ゲームの成功率の間の関係性はわからないため,
繰り返しゲームと考えても次のゲームへ引き継げる情報がなかったため将来期待利得を考慮しなかったが,
「告発する約束・評判ゲーム」においては先に述べた関係式が得られるため,
将来期待利得を考慮する必要がある.

商取引で「成功」が報告された場合の$ seller $と$ buyer $の将来期待利得を$ \epsilon^{seller}, \epsilon^{seller} $とし,
「失敗」が報告された場合の将来的な期待利得を$ \lambda^{seller}, \lambda^{buyer} $とおくと,
「告発する約束・評判ゲーム」のゲーム木と非協力戦略型ゲームの表は次のように書き換えられる.

また,本稿では,これらについて$ \epsilon^{player} > \lambda^{player} $と仮定する.
