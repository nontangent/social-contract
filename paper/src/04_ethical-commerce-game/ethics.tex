\section{倫理とは}
倫理とは、集団の利益のために直結する限定合理性である。


「不正防止の不可能性」より,「商取引ゲーム」においては不正を防止するインセンティブ設計ができないことがわかった.しかし,この理論上の商取引に反して,実社会の商取引においては不正行為がある程度、抑制されている.これは実社会の商取引を行う人々が,自身の利益を最大化するような利己的な行動規範に従って戦略を選んでおらず,完全な「商取引ゲーム」を行っていないためだと考えられる.本稿では、実社会の商取引において「泣き寝入り」と呼ばれるような$buyer$が不正にあった場合に「成功」を報告する戦略があまり取られていないことに着目し、$buyer$が不正にあった場合に必ず「失敗」を報告する行動規範を倫理として定義し、この倫理に従うプレイヤーのみで行われる「商取引ゲーム」を「倫理ある商取引ゲーム」とする.


\section{倫理ある行動}
% 「不正防止の不可能性」より,「商取引ゲーム」においては不正を防止するインセンティブ設計ができないことがわかった.しかし,この理論上の商取引に反して,実社会の商取引においては不正行為がある程度抑制されている.これは実社会の商取引を行う人々が,自身の利益を最大化するような利己的な行動規範に従って戦略を選んでおらず,完全な「商取引ゲーム」を行っていないためだと考えられる.そこで、本稿ではそのような行動規範を倫理と定義し、利己的な行動規範との違いについて論じる.

% 利己的な行動規範とは,相手が各戦略をとる主観的な確率とその際の自身の利得から自身の各戦略の期待利得を算出し,それが高い方の戦略を選択する行動規範である.各利得は「商取引システム」から一意に決定されるため,「倫理」と利己的な行動規範の差異は相手が各戦略をとる主観的な確率にあると考えられる.つまりは相手が特定の戦略をとる確率を通常よりも低く(もしくは高く)見積もっていることになる.そのため倫理が具体的にどのように利己的な行動規範と違うのかを知るためには、$seller$と$buyer$のとりえる各戦略について考える必要がある.
%
% 「商取引ゲーム」においては$seller$の戦略は2通りあり$buyer$の戦略は4通りあった.これらの戦略は「$seller$が商品を渡すか否か」と「$buyer$が商品の受け取りを報告するか否か」,「$buyer$が不正にあった場合に告発するか否か」の3つの観点から2分できる.そこで,それらの観点で2分した片方の戦略群を相手が選ぶ可能性が0であるとした場合について考える.また,この主観的な確率が0であるとみなす戦略をそのプレイヤーはとりえないものとして考える.本稿では、実社会の商取引において「泣き寝入り」と呼ばれるような戦略があまり取られていないことに着目し、$buyer$が不正にあった場合に必ず「失敗」を報告する行動規範を倫理として定義する.
