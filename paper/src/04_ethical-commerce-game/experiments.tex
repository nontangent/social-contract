\section{実験方法}
提案手法を用いることで不正が防止されるかどうかを検証するためにコンピューターシミュレーションを用いた実験を行う.この実験では$ (s^{seller_1}, s^{seller}_2) $と$ (s^{buyer}_1, s^{buyer}_2, s^{buyer}_3, s^{buyer}_4) $から任意の戦略組をもつエージェント(全8通り)を10機用意して下記の試行を繰り返す.試行ごとに過去100回分の「真の商取引の成功率(TrueSuccessRate)」と「商取引システム」に「報告された成功率(ReportedSuccessRate)」を記録する.

①10機のエージェントからランダムに$ seller $と$ buyer $を決定し「商取引ゲーム」を行う.
②残高の変化量の組$ (r^{seller}_{success}, r^{buyer}_{success}, r^{seller}_{failure}, r^{buyer}_{failure}) $を算出し,「失敗」が報告された場合に$ seller $と$ buyer $の残高が0未満にならないかを判定する.ここで0未満になる場合は再度,①からやり直し,100回連続で
③$ seller $と$ buyer $のエージェントは戦略から「商取引システム」に報告される結果を決定する.
④「商取引システム」は報告された結果に応じてすべてのエージェントの残高を操作する.
⑤試行回数が10000回になるまで①から④を繰り返す.

\subsection{「 倫理ある商取引ゲーム」での実験}
「倫理ある商取引ゲーム」では$ seller $と$ buyer $がとりえる戦略は$ (s^{seller_1}, s^{seller}_2) $と$ (s^{buyer}_1, s^{buyer}_3 $のみであるため,これらを組み合わせた4パターンのエージェントを全体で10機用意して実験を行う.この10機の4パターンでの構成は268通りあるので,その全て場合で1回づつ実験を行う.

\subsection{「商取引ゲーム」での実験}
「倫理ある商取引ゲーム」では$ seller $と$ buyer $がすべての戦略をとりえるため,8パターンのエージェントを全体で10機用意して実験を行う.この10機の4パターンでの構成は19960通りあるので,モンテカルロ法を用いてランダムに重複ありの1000パターンの構成を生成し実験を行う.
