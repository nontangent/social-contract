\section{結論}
   本稿では,商取引において不正行為を防止するためには,「行動観察不可の条件」を満たした「第3者に依存しない仲介システム」が存在している必要があることを論じた.その上で,そのシステムが存在するのかを検証するために,下記の条件を付随した「商取引システム」と,それを仲介とした「商取引ゲーム」を定義して,商取引で不正行為を防止できるインセンティブ設計が可能かを確かめた.

  \begin{itemize}
    \item システムは$buyer$によって報告された商取引の結果を観察できる.
    \item システムは$seller$と$buyer$がどの戦略を選んだかはわからない.
    \item 商取引に参加する$player$は合理的に(利得の期待値が最も高い)戦略を決定する.
    \item システム内には追跡可能な通貨が存在しており,商取引にはその通貨が用いられる.
    \item システムからは$player$の通貨の保有量を操作することができる.
  \end{itemize}

  その結果として,不正行為が防止されるための2つの戦略組のいずれかに$seller$と$buyer$の戦略を帰着させるような利得の組を商取引システムから決定することはできないことを証明した.これはつまり,人々は合理的であるという仮定の上で成り立つ「商取引ゲーム」において,不正行為を防止することが不可能であることを意味する.