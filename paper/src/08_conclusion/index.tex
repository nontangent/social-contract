\chapter{結論}
\section{本論のまとめ}
補題1〜3の検証より、下記の仮定の上で、強制執行力が集団の外部に存在しない場合であっても、
一部の成員が誠実に振る舞っていれば、成員達が事前に合意された約束が必ず履行される状態に至ることを示した。
また、ここまでの検証の仮定から歴史と社会指標、各成員の振る舞いについて次のように記すことができる。
誠実な成員の人数と社会契約の成立の関係性については全勝の図\ref{compex-system-002}が示す通りである。
ここから読み取る限り、誠実な成員の人数が過半数を超える場合、社会契約は必ず成功すると考えられる。

\subsection{仮定}
\begin{description}
  \item[仮定1] 各成員は、他の成員の行動を観察できない。  
  \item[仮定1] 送信されたすべてのメッセージは正しく到達する
  \item[仮定2] メッセージの受信者は誰が送信したのかわかる
  \item[仮定3] メッセージが届かないことを検知できる
  \item[仮定4] 誠実な成員の署名は偽造できず、署名されたメッセージの内容が変更されても、それを検知することができる。
  \item[仮定5] 誰でも成員の署名の信憑性を検証することができる。
  \item[仮定5] 各成員は、同期された時刻を知ることができる。 
  \item[仮定2] 各成員は、成員の人数について、事前に合意している。
  \item[仮定3] 各成員は、成員の社会指標の決定方法について、事前に合意している。
  \item[仮定4] 各成員は、各時刻の約諾者と報告者について、事前に合意している。
\end{description}

\subsection{歴史}
社会契約に最低限必要となる歴史とは、各時刻$t$における約束の記録の集合である。
約束の記録とは報告された時刻、約諾者($promisor$)、報告者($reporter$)、約束の結果(「履行」もしくは「反故」)の4つの情報からなるものである。

\subsection{社会指標}
社会指標とは、初期の評判スコアと過去の歴史から導き出される評判の指標である。
節で導き出した条件を満たすとき、告発する成員が一定数以上いれば約束を保護にする成員の社会指標を下げることができる。

\subsection{成員の振る舞い}
前の章のをいい感じに!!

\section{本研究の課題}
本設では、本研究に残された3つの課題について取り上げる。

第1に、導き出した誠実な成員と社会契約の成立の関係性はヒューリスティックな解にすぎないことである。
本研究の補題2と3の検証のために扱った実験は、
8種類のエージェント8体の構成は順序も含めて$16,777,216$($=8^8$)通り考えられるうちの
$64,000$件のサンプルから導き出されるものに過ぎない。
そのため誠実な成員が過半数以上いる場合に社会契約が成立するのかについては慎重な議論が必要である。
仮に厳密な誠実な成員と社会契約の成立の関係性を知ろうとするならば、
全ての可能性を総当りする大規模な実験を行うか、理論的な解明が必要となるだろう。

第2に、補題3の検証で用いたビザンチン将軍問題の解法が現実的ではない点が挙げられる。
本研究では補題2と補題3の検証結果の類似性を示すために、
成員の人数が$f+1$($f$は誠実でない成員の人数)の場合にビザンチンフォールトトレラントの達成される
ビザンチン将軍問題の初期の署名付きの解法\cite{lamport1982}を用いたが、
この解法は同期的なシステムを前提としており、非同期システムにおいては成立しないことが知られている。\cite{fischer1985}
そのため、より現実的な想定をするならば、pBFTのようなアルゴリズムを用いて成員の振る舞いを記述するべきである。\cite{castro1999}
こちらは成員の人数が$3f+1$以上の場合にビザンチンフォールトトレラントが達成されるため、
実験のデザインには注意が必要となるだろう。

第3に、成員の人数が時間経過とともに変動する場合についてモデルを拡張する必要がある。
現実の社会に目を向けたとき集団の人数は時間とともに変化していくものであるが、
本研究のモデルはこうした集団の人数の変化を考慮されていない。
これを実現するためには、成員の人数や各時刻の約諾者と報告者の組み合わせなど、
いくつかの事前の合意事項を更新する必要がある。
分散システムにおけるSybil Attack\cite{douceur2002}のように、
複数の架空の成員を追加することで社会契約の成立を妨げる攻撃が想定されるため、
追加される成員の評判スコアや最低信頼度については慎重な議論が必要である。

これらは「本研究の動機(\ref{motivation}節)」で述べたような
社会契約のインターネットへの適用という目的のためには避けては通れない課題である。

\section{今後の研究}
今後の研究としては先に上げた3つの課題の解決が必要とされる。
また、仮にはじめの2つの課題が解決され、
誠実でない誠意の数$f$に対して成員が$2f+1$以上存在する場合に約束が必ず履行される状態に至り、
pBFT\cite{castro1999}のような$3f+1$以上の成員を必要とするアルゴリズムでも本研究と同じ結果が得られるのであれば、
時間経過とともに$f$が増加する場合に$3f+1$以下の人数でもビザンチンフォールトトレラントを達成できる可能性がある。
これは、本研究の補題3の検証が示す通り、$2f+1$以上の場合にビザンチン故障の原因になりうる成員が排除されるためである。

例えば8人中2人の成員が誠実でない場合、$3f+1$(と同時に$2f+1$)を満たしているため、
時間経過とともに誠実でない2人の成員は排除され、必ずpBFTのアルゴリズムが守られる状態に至る。
その後、誠実だった6人の成員のうち1人が不誠実になったとしても、
これは$3f+1$を満たしているため、ビザンチンフォールトトレラントが達成されて、その1人も排除されることになる。
結果的に見れば8人中3人が不誠実な成員という$3f+1$に反する状況下でもビザンチンフォールトトレラントを達成することができることとなる。

無論、検証するまでこれが確かであるかはわからないが、
これが可能であるならば第3の課題のような新規に成員が増える場合についても役立ちそうである。

最後に、今回用いたビザンチン将軍問題の解法をはじめ、
計算機科学における分散システムの知見は社会契約のモデリングに大いに役立つと考えられる。
(EigenTrust\cite{kamvar2003}などは非常に親しい概念を取り扱っている。)
また、逆にこれまで社会契約の議論の中で用いられてきた様々なアプローチは、
計算機科学における分散システムの研究に何らかの形で影響を与えるかもしれない。
そういった今後この両分野の関係性について、より詳細な研究が必要となるだろう。