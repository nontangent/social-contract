\chapter{結論}
\section{本論のまとめ}
\begin{description}
  \item[仮定0] 外部に強制執行力が存在しない。 
  \item[仮定1] 各成員は、他の成員の行動を観察できない。
  \item[仮定2] 各成員は、成員の人数について、事前に合意している。
  \item[仮定3] 各成員は、成員の社会指標について、事前に合意している。
  \item[仮定4] 各成員は、各時刻$t$の約諾者と報告者について、事前に合意している。
  \item[仮定5] 各成員は、同期された時刻$t$を知ることができる。   
\end{description}

強制執行力が存在せず、各成員が他の成員の行動を観察できない場合であっても、
上記の5つの仮定の上で、
歴史と社会指標に基づいた成員の振る舞いを行えば、
時間の進行とともに合意された約束を履行しない成員は排除され、
事前に合意された約束が守られる状態に至る(社会契約が成立する)ことがわかった。
誠実な成員の人数と社会契約の成立の関係性については図が示す限りである。

また、ここまでの検証の仮定から歴史と社会指標、各成員の振る舞いについて次のように記すことができる。

\subsection{歴史}
社会契約に最低限必要となる歴史とは、各時刻$t$における約束の記録の集合である。

\subsubsection{約束の記録}
\begin{description}
  \item[時刻] … 約束の記録が報告された時刻
  \item[約諾者] … 約束を履行する成員のID
  \item[報告者] … 約束の記録を報告する成員のID 
  \item[結果] … 約束が履行されたか反故にされたかの結果(「成功」もしくは「失敗」とする)
\end{description}

\subsection{社会指標}
社会指標とは、なんだ!!?

\subsubsection{社会指標の決定アルゴリズム}
評判システムを式に!!

\subsection{成員の振る舞い}
前の章のをいい感じに!!

\section{本論の課題}
実際の社会を想定した場合、成員の人数が増えることが考えられる。
本論で示したモデルでは定数の場合しか示せていない。
また、強制執行力が存在する場合、検証2をヒューリスティックなカタチでして検証できない。
もっと3k+1なら安全!!的なことを言える証明をつけるか、すごいパソコン使って全通り試したぜ!!的なことが言いたい、、、

\section{今後の研究}
たぶん検証2はエージェントの種類の構成次第の複雑系なんで、解明できる自身がない。
すごいパソコンで全通り探索したい。
検証2の結果だと3f+1くらいの場合は確実に約束が遵守されそうだし、
署名なしのビザンチン将軍問題の解でも6人以上の場合は同じ結果になりそう。
検証3はPBFTとかのアルゴリズムにもできるかもしれない。
$f$の評判スコアを落として排除することができるので、
時刻$t$に置いて$3f+1$が守られてれば時刻$t+1$で$f$が増えても耐えられるアルゴリズムが組めるかもしれない。
あと、今回、集団の人数が一定の場合しかやってないからやってないから人数が変化する場合も考えたい。
たぶん新規参加者の評判スコアをどうするかが論点になる。