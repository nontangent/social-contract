\chapter{結論}
\section{本論のまとめ}
本研究では、外部の強制執行力が存在しない場合に社会契約が成立しうるかという問いに対して、3つの補題に分けてそれぞれを検証した。
第1の補題については、ゲーム理論的なアプローチによって、完全に合理的な成員のみで構成される集団において社会契約が成立しないことを示した。
次の補題については、進化ゲーム理論的なアプローチによって、一部の成員が限定合理的であれば社会契約が成立しうることを示した。
そして最後の補題については、ビザンチン将軍問題の署名付きの解法を用いることで補題2の強制執行力を集団の内部で自己組織化することができることを示した。
これらの検証の結果から、下記の仮定の上で一部の成員が誠実に振る舞っていれば社会契約が成立することがわかる。
このときの誠実な成員の人数と社会契約の成立の関係性については前章の図\ref{compex-system-002}が示す通りであり、
ここから誠実な成員が半数を超える場合には必ず約束が履行される状態に至っていたことがわかる。
また、履歴と社会指標、各成員の振る舞いについてはそれぞれ下記のように記すことができる。

\subsection{仮定}
\begin{enumerate}
  \item 報告者($reporter$)以外の成員は約諾者($promisor$)の行動を観察できない。  
  \item 各成員は同期された時刻を知ることができる。 
  \item 各成員は集団の人数について事前に合意している。
  \item 各成員は成員の社会指標の決定方法について事前に合意している。
  \item 各成員は各時刻の約諾者と報告者について事前に合意している。
  \item 送信されたすべてのメッセージは正しく到達する
  \item メッセージの受信者は誰が送信したのかわかる
  \item 各成員はメッセージが届かないことを検知できる
  \item 誠実な成員の署名は偽造できず、署名されたメッセージの内容が変更されても、それを検知することができる。
  \item 誰でも成員の署名の信憑性を検証することができる。
\end{enumerate}

\subsection{履歴}
社会契約に最低限必要となる履歴とは、各時刻$t$における約束の記録の集合である。
約束の記録とは報告された時刻、約諾者($promisor$)、報告者($reporter$)、約束の結果(「履行」もしくは「反故」)の4つの情報からなるものである。

\subsection{社会指標}
社会指標とは、初期の評判スコアと過去の履歴から導き出される評判の指標である。
\ref{conditionByTrustScore}節で導き出した条件を満たすとき、告発する成員が一定数以上いれば約束を反故にする成員の社会指標を下げることができる。

\subsection{成員の振る舞い}
  \begin{itemize}
    \item 事前の合意に基づいて、全ての成員の人数$n$を決定する。
    \item 事前の合意に基づいて、各成員の初期の「評判スコア」$b_i, ..., b_n$を定義する。
    \item 事前の合意に基づいて、全ての成員が$promisor$と$reporter$のそれぞれの役割で総当りする周期を定義する。(周期の長さは$ n * (n-1)$)
    \item 事前の合意に基づいて、「評判システム」を用意する。
    \item 事前の合意に基づいて、約束の内容を定義する。
    \item 各時刻$t$において、「時刻tにおける成員の振る舞い」を上から順に実行する。
  \end{itemize}

\subsubsection{時刻tにおける成員の振る舞い}
\begin{enumerate}
  \item 事前に合意した周期から$promisor$と$reporter$を決定する。
  \item $reporter$は時刻$t-n(n-1)$に$promisor$と交わした約束の結果を決定する。$t \leq n(n-1)$の場合、「成功」とする。
  \item $V_i=\emptyset$として初期化する。($\emptyset$は空集合)
  \item $reporter$は約束の結果を全ての成員に署名して送る。
  \item 各$i$について、 
  \begin{enumerate}
    \item もし$member_i$が$v:0$という形式のメッセージを受け取り、まだ何の報告も受けていない場合は、
    \begin{enumerate}
      \item $member_i$は$ V_i$を${v}$にする。 
      \item $member_i$は他のすべての成員にメッセージ$v:0:i$を送ります。
    \end{enumerate}
    \item もし$member_i$が$v:0:j_1:...:j_k$という形式のメッセージを受け取り、$v$が集合$V_i$に入っていない場合は
    \begin{enumerate}
      \item $member_i$は$v$を$V_i$に追加する。
      \item もし$k<m$であれば、$member_i$は$j_1, ..., j_k$以外のすべての副官に$v:0:j_1:...:j_k:i$というメッセージを送る。
    \end{enumerate}
  \end{enumerate}
  \item 各$i$について、$member_i$がこれ以上メッセージを受け取らない場合、$member_i$は$choice(V_i)$を報告された結果とする。
  \item 各$i$について、$member_i$は報告された結果を自身の「評判システム」に記録する。
  \item $reporter$は$promisor$と新たな約束を交わす。
\end{enumerate}

\subsubsection{関数 $choice(V)$}
関数$choice(V)$は集合$V$を引数にとって約束の結果(「履行」か「反故」)を返す関数である。
\begin{enumerate}
  \item[1.] もし集合に単一の約束の結果$v$しか存在しなければ、$choice(V) = v$とする。
  \item[2.] $choice(\emptyset) = 反故$とする。$\emptyset$は空集合。
  \item[3.] もし集合に約束の結果が存在するとき、$choice(V) = 反故$とする。
\end{enumerate}

\section{本研究の課題}
本研究には次に挙げる3つの課題がある。

第1に、導き出した誠実な成員と社会契約の成立の関係性はヒューリスティックな解にすぎない点が挙げられる。
本研究の補題2と3の検証のために扱った実験は、
8種類のエージェント8体の構成は順序も含めて$16,777,216$($=8^8$)通り考えられるうちの
$64,000$件のサンプルから導き出されるものに過ぎない。
そのため誠実な成員が過半数以上いる場合に社会契約が成立するのかについては慎重な議論が必要である。
仮に厳密な誠実な成員と社会契約の成立の関係性を知ろうとするならば、
全ての可能性を総当りする大規模な実験を行うか、理論的な解明が必要となるだろう。

第2に、補題3の検証で用いたビザンチン将軍問題の解法が現実的でない点が挙げられる。
本研究では補題2と補題3の検証結果の類似性を示すために、
成員の人数が$f+1$($f$は誠実でない成員の人数)の場合にビザンチンフォールトトレランスの達成される
ビザンチン将軍問題の初期の署名付きの解法\cite{lamport1982}を用いたが、
この解法は同期的なシステムを前提としており、非同期システムにおいては成立しないことが知られている。\cite{fischer1985}
そのため、より現実的な想定をするならば、pBFTのようなアルゴリズムを用いて成員の振る舞いを記述するべきである。\cite{castro1999}
こちらは成員の人数が$3f+1$以上の場合にビザンチンフォールトトレランスが達成されるため、
実験のデザインには注意が必要となるだろう。

第3に、成員の人数が時間経過とともに変動する場合についてモデルを拡張する必要がある。
現実の社会に目を向けたとき集団の人数は時間とともに変化していくものであるが、
本研究のモデルはこうした集団の人数の変化を考慮されていない。
これを実現するためには、成員の人数や各時刻の約諾者と報告者の組み合わせなど、
いくつかの事前の合意事項を更新する必要がある。
これには分散システムにおけるSybil Attack\cite{douceur2002}のように、
複数の架空の成員を追加することで社会契約の成立を妨げる攻撃が想定されるため、
追加される成員の評判スコアや最低信頼度について慎重に議論する必要がある。

これらは「本研究の動機(\ref{motivation}節)」で述べたような
社会契約のインターネットへの適用という目的のためには避けては通れない課題である。

\section{今後の研究}
今後の研究としてはこれら3つの課題の解決が必要とされるが、とりわけはじめの2つの課題が重要であると考える。
仮にその2つの課題が解決されるならば、
時間経過とともに$f$が増加する場合に$3f+1$以下の人数でもビザンチンフォールトトレランスを達成できる可能性がある。
もし誠実でない誠意の数$f$に対して成員が$2f+1$以上存在する場合に約束が必ず履行される状態に至り、
pBFT\cite{castro1999}のような$3f+1$以上の成員を必要とするアルゴリズムでも本研究と同じ結果が得られるのであれば、
成員の人数が$3f+1$以上のとき、本研究の補題3の検証結果と同様に、ビザンチン故障の原因になりうる成員が排除できるためである。

例えば8人中2人の成員が誠実でない場合、$3f+1$(と同時に$2f+1$)を満たしているため、
時間経過とともに誠実でない2人の成員は排除され、必ずPBFTのアルゴリズムが守られる状態に至る。
その後、誠実だった6人の成員のうち1人が不誠実になったとしても、
これは$3f+1$を満たしているため、ビザンチンフォールトトレランスが達成されて、その1人も排除されることになる。
結果的に見れば8人中3人が不誠実な成員という$3f+1$に反する状況下でもビザンチンフォールトトレランスを達成することができることとなる。

無論、検証するまでこれが確かであるかはわからないが、
これが可能であるならば第3の課題のような新規に成員が増える場合についても何らかの解決の糸口になりそうである。

また、今回用いたビザンチン将軍問題の解法をはじめとした
計算機科学における分散システムの知見は社会契約のモデリングに大いに役立つと考えられる。
(EigenTrust\cite{kamvar2003}などは非常に近しい概念を取り扱っている。)
また、逆にこれまで社会契約の議論の中で用いられてきた様々なアプローチは、
計算機科学における分散システムの研究に何らかの形で影響を与えるかもしれない。
そういった今後この両分野の関係性について、より詳細な研究が必要となるだろう。
