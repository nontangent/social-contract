% ビザンチン将軍問題に基づく分散型評判システムとしての社会契約

\chapter{仮説と検証方法}
\section{本論の仮説}
\label{hypothesis}
筆者は、先の問について、外部の強制執行力が存在しないとき、一部の成員が誠実に振る舞っていれば社会契約は成立すると考えている。
本論では、この仮説を示すために下記の3つ仮説を補題として扱い順に検証する。

\def\firstLemma{外部に強制執行力が存在するとき、全ての成員が完全に合理的ならば社会契約は成立しない。}
\def\secondLemma{外部に強制執行力が存在するとき、一部の成員が限定合理的ならば社会契約は成立する。}
\def\thirdLemma{成員の振る舞いによって補題2の強制執行力を自己組織化することができる。}

\begin{description}
  \item[補題1] \firstLemma
  \item[補題2] \secondLemma
  \item[補題3] \thirdLemma
\end{description}

\section{検証方法}
本論では、補題1〜3を順に検証することで、仮説を示す。その手順について概要をここで述べる。

まず、成員の行動を観察できない外部の強制執行力を「評判システム」と定義した上で議論を進め、補題1を示す。
「評判システム」とは報告された約束の結果に基づいて各成員の評判スコアを決定するシステムである。
また、そのシステムを用いた任意の約束を結ぶ2人の「約束・評判ゲーム」について考える。
これは一方($promisor$)が約諾し、もう一方($reporter$)がその約束の結果(「成功」か「失敗」)を「評判システム」に報告する非協力戦略型ゲームである。
このゲームに参加するプレイヤーが完全に合理的な場合、両者がとりうる各戦略の利得を比較することで、
彼らの約束が真に成功する($promisor$が約束を履行し、$reporter$が「成功」を報告する)条件を導く。
全ての成員が完全に合理的な場合、「評判システム」が報告された約束の結果から、
その条件を満たす評判スコアを決定することができないことを示す。

次に、補題1を踏まえて、全ての成員が約束を反故にされた場合に「失敗」を報告する「告発する約束・評判ゲーム」について考えることで補題2を示す。
そして、補題1と同様に、成員がとりうる各戦略の利得を比較することで、この「告発する約束・評判ゲーム」において、彼らの約束が真に成功する条件を導く。
「告発する約束・評判ゲーム」においては、「評判システム」が報告された約束の結果(「成功」か「失敗」)から、その条件を満たす評判スコアを決定することができることを示す。
その条件を満たす「評判システム」の詳細を定義し、
戦略の限定されない通常の「約束・評判ゲーム」において、定義した評判システムが機能するかマルチエージェントシミュレーションを用いて検証する。
その結果から、集団を構成する成員達の性質によっては、全ての約束が成功する状態に至ることを示す。

最後に、各成員が「評判システム」を所有している場合について考えることで、補題3を示す。
ビザンチン将軍問題の署名付きの解決策を用いて、「評判システム」を各成員の振る舞いによる分散システムとして設計可能であることを示す。
補題2の条件に基づいて実装された「評判システム」を同様に各成員の振る舞いとして定義し、マルチエージェントシミュレーションを用いて検証する。
その結果から、外部の強制執行力が存在しない場合でも、成員の振る舞いによって強制執行力を自己組織化させることが可能であり、
集団を構成する成員達の性質によっては、全ての約束が成功する状態に至ることがわかる。

また、補題3の実験結果から、「評判システム」が成員達の振る舞いによって自己組織化された分散システムとして機能し、
集団を構成する成員達の性質によっては、「約束-評判ゲーム」で全ての約束が成功する状態に至ることが示される。
これにより、先の仮説が立証され、社会契約の成立に必要となる履歴の定義や社会指標の決定方法、
具体的な成員の振る舞い、その振る舞いに従う成員の人数と社会契約の成立の関係性を明らかにすることができる。
