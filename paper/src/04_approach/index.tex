\chapter{仮説と検証方法}
\section{本論の仮説}
外部の強制執行力が存在しない場合であっても、
集団を構成する成員の性質によっては強制執行力を生み出すことが可能であり、
それによって成員たちに合意した約束を遵守させることが可能である。


\section{仮説の検証方法}
本論では、3つの補題の仮説検証を行うことで、先の仮説を示す。

第1に、全ての成員が完全に合理的である場合、
成員の振る舞いによって決定された歴史から各成員の社会指標を算出できる外部の強制執行力が、
成員達に約束を遵守させるようなインセンティブ設計をできないことを示す。
私達は、この検証のために、任意の約束を商取引契約として結ぶ売り手と買い手の商取引ついて考える。
具体的には、売り手が報告した商取引の結果を歴史として記録し、各成員の社会指標である通貨保有量を決定することができる
外部の強制執行力としての「商取引システム」の存在を仮定する。
また、この「商取引システム」を用いて行われる商取引を「商取引ゲーム」という非協力戦略型ゲームとしてモデリングする。
そして、買い手と売り手の各戦略の利得を比較することで、
全ての成員が合理的である場合に「商取引システム」から報告された商取引の結果に基づいて不正が防止されるような買い手と売り手の通貨保有量を決定することが不可能であることを示す。

第2に、一部あるいは全部の成員が限定合理的である場合、
成員の振る舞いによって決定された歴史から各成員の社会指標を算出できる外部の強制執行力が、
合理的な成員達に約束を遵守させるようなインセンティブ設計をできることを示す。
この検証のために、我々は先の「商取引ゲーム」において報復的な戦略(買い手が商取引契約を履行しなかった場合に「失敗」を報告する戦略)をとる成員のみによって行われる
「倫理ある商取引ゲーム」について考える。
はじめの検証と同じように、この「倫理ある商取引ゲーム」における買い手と売り手の各戦略の利得を比較することで、
このモデルにおいては「商取引システム」から報告された商取引の結果に基づいて不正を防止できるような買い手と売り手の通貨保有量を決定することが可能であることを示す。
また、この条件に基づいて各誠意の通貨保有量を操作する商取引システムを設計し、
本来の「商取引ゲーム」に適用してマルチエージェントシミュレーションを用いた実験を行うことで、
全ての成員が先に述べた報復的な戦略を取る場合でなくても、一部あるいは全部の成員が報復的な戦略をとっていれば、不正が防止されうることを示す。

第3に、外部の強制執行力が存在しない場合でも、一部あるいは全部の成員が限定合理的である場合、
各成員の振る舞いのみによって成員達に約束を遵守させる強制執行力を生じさせることが可能であることを示す。
この検証のために、我々は先の検証で外部の執行力として存在が仮定されていた「商取引システム」を
集団が構成する複雑系の中で各誠意の振る舞いによって自己組織化されたシステムとして設計し、第2の検証と同様の実験を行う。
「商取引システム」を各成員の振る舞いによって自己組織化されたシステムとして再現することができ、
一部あるいは全部の成員が報復的な戦略をとっていれば、不正が防止されうることを示す。

これら3つの補題の検証により、外部の強制執行力が存在しない場合であっても、
集団を構成する成員の性質によっては強制執行力を生み出すことが可能であり、
それによって成員たちに合意した約束を遵守させることが可能だとわかる。
また、問題提起の要件で述べた社会契約の成立に必要となる歴史の定義や社会指標の計算方法、
具体的な成員の振る舞いがそれぞれ明確になる。