\documentclass[a4j]{ujreport}
\usepackage{multirow}
\usepackage{amsmath,amssymb}
\usepackage[T1]{fontenc}
\usepackage[dvipdfmx]{graphicx}
% \usepackage[dvipdfmx]{hyperref}
% \usepackage{pxjahyper}
% \usepackage{comment}
\usepackage{tabularx}
% \usepackage{listliketab}
% \usepackage[longnamesfirst]{natbib}
% \usepackage[dvipdfmx]{graphics}
% \usepackage[dvipdfmx]{graphicx}
% \usepackage[dvipdfmx]{color}
% \usepackage{subfigure}
% \usepackage{alltt}
% \usepackage{here}
% \usepackage{afterpage}
% \usepackage{./sty/ncodeline}
% \usepackage{enumerate}
% \usepackage{footnote}
% \usepackage{amssymb}
% \usepackage{array}

% A4  size: 297mm*210mm %1pt = 0.35mm
\setlength{\topmargin}{-3.4mm} % 10pt 25.4mm - 3.4mm = 22mm
\setlength{\oddsidemargin}{-0.4mm} % 25.4mm - 0.4mm = 25mm
\setlength{\evensidemargin}{-0.4mm} % 25.4mm - 0.4mm = 25mm
\setlength{\textheight}{231mm} % 660pt % original is 225.75mm 645pt
\setlength{\textwidth}{160mm} % 457pt

% 各種ページに占める割合
\renewcommand{\topfraction}{.99}
\renewcommand{\textfraction}{.0}
\renewcommand{\floatpagefraction}{.99}

% 参考文献の表記を修正
\renewcommand{\bibname}{参考文献}

% 各ページ右上に章のタイトルを表示する
\usepackage{fancyhdr}
\pagestyle{fancy}
\lhead[]{}

% よくわからないおまじない
\makeatletter
\def\chaptermark#1{\markboth {\ifnum \c@secnumdepth>\m@ne
\@chapapp\ \thechapter \@chappos\ \fi #1}{}}
\def\lst@lettertrue{\let\lst@ifletter\iffalse}
\makeatother

% メタデータ
\def\title{「商取引ゲーム」をベースにした\\計算可能な複雑系としての社会契約}
\def\etitle{Social Contract as a Computable Comlex System based on Commerce Game}
\def\author{宮元 眺}
\def\eauthor{Nozomu Miyamoto}
\def\dept{慶應義塾大学 環境情報学部}
\def\edept{Keio University Faculty of Environment and Information Studies}
\date{\today}

\begin{document}

% 表紙
\begin{titlepage}
  \begin{center}
    \begin{large}
      卒業論文   2021年度(令和3年)\\
      \vspace{24pt}
      \title
    \end{large}
  \end{center}
  \vspace{40em}
  \begin{flushright}
    \large \dept\\
    \author
  \end{flushright}
\end{titlepage}

% 概要
\chapter*{謝辞}
本研究を進めるにあたり、2016年度秋学期から所属していた研究軍団NECO及び
RG(村井・徳田・中村・楠本・高汐・バンミーター・植原・三次・中澤・手塚・武田合同研究会)の皆様、
慶應SFCの教員の皆様、大学進学を支援してくださった家族に感謝します。
中でも早稲田大学 大学院経営管理研究科 教授 斉藤賢爾氏には大変お世話になりました。
本研究が全く問題意識をはっきりとしない頃から議論にお付き合いいただいたことは勿論、
斉藤氏が著書「信用の新世紀 ブロックチェーン後の未来」\cite{saito2017}の中で描かれるこの社会の未来像は
この研究の大きなモチベーションになりました。
彼のデジタルマネーに関する思想に出会わなければ、
本研究における問題意識を抱くこともなく社会契約を研究対象として選ぶこともなかったと言えます。
また、代々KGLとしてNECOの運営を担ってくださった阿部涼介氏、菅藤 佑太氏、島津 翔太氏、渡邉 聡紀氏にも感謝いたします。
特に慶應義塾大学 政策・メディア研究科 博士課程の阿部涼介氏は自分にも他人にも厳しい方でしたが、
加えて、彼の研究への姿勢から多くを学ぶことができました。
日頃より、RGの運営や講義、研究発表でお世話になっている
慶應義塾大学教授 村井純博士、同学部教授 中村修博士、同学部教授 楠本博之博士、同学部教授 高汐一紀博士、
同学部教授 Rodney D.Van Meter III 博士、同学部准教授 植原啓介博士、
同学部教授 三次仁博士、同学部教授 中澤仁博士、同学部教授 武田圭史博士、
同大学政策・メディア研究 科特任准教授 佐藤雅明博士、同大学政策・メディア研究科特任教授 鈴木茂哉博士に感謝いたします。
そうした運営に携わってくださったRG Coordinatorの方々もありがとうございます。
最後に、慶應SFCという素敵なキャンパスでの大学生活を支えてくださった
教員の皆さまや友人、家族に感謝しております。


\tableofcontents\thispagestyle{plain}

% 序論
\chapter*{謝辞}
本研究を進めるにあたり、2016年度秋学期から所属していた研究軍団NECO及び
RG(村井・徳田・中村・楠本・高汐・バンミーター・植原・三次・中澤・手塚・武田合同研究会)の皆様、
慶應SFCの教員の皆様、大学進学を支援してくださった家族に感謝します。
中でも早稲田大学 大学院経営管理研究科 教授 斉藤賢爾氏には大変お世話になりました。
本研究が全く問題意識をはっきりとしない頃から議論にお付き合いいただいたことは勿論、
斉藤氏が著書「信用の新世紀 ブロックチェーン後の未来」\cite{saito2017}の中で描かれるこの社会の未来像は
この研究の大きなモチベーションになりました。
彼のデジタルマネーに関する思想に出会わなければ、
本研究における問題意識を抱くこともなく社会契約を研究対象として選ぶこともなかったと言えます。
また、代々KGLとしてNECOの運営を担ってくださった阿部涼介氏、菅藤 佑太氏、島津 翔太氏、渡邉 聡紀氏にも感謝いたします。
特に慶應義塾大学 政策・メディア研究科 博士課程の阿部涼介氏は自分にも他人にも厳しい方でしたが、
加えて、彼の研究への姿勢から多くを学ぶことができました。
日頃より、RGの運営や講義、研究発表でお世話になっている
慶應義塾大学教授 村井純博士、同学部教授 中村修博士、同学部教授 楠本博之博士、同学部教授 高汐一紀博士、
同学部教授 Rodney D.Van Meter III 博士、同学部准教授 植原啓介博士、
同学部教授 三次仁博士、同学部教授 中澤仁博士、同学部教授 武田圭史博士、
同大学政策・メディア研究 科特任准教授 佐藤雅明博士、同大学政策・メディア研究科特任教授 鈴木茂哉博士に感謝いたします。
そうした運営に携わってくださったRG Coordinatorの方々もありがとうございます。
最後に、慶應SFCという素敵なキャンパスでの大学生活を支えてくださった
教員の皆さまや友人、家族に感謝しております。

\chapter*{謝辞}
本研究を進めるにあたり、2016年度秋学期から所属していた研究軍団NECO及び
RG(村井・徳田・中村・楠本・高汐・バンミーター・植原・三次・中澤・手塚・武田合同研究会)の皆様、
慶應SFCの教員の皆様、大学進学を支援してくださった家族に感謝します。
中でも早稲田大学 大学院経営管理研究科 教授 斉藤賢爾氏には大変お世話になりました。
本研究が全く問題意識をはっきりとしない頃から議論にお付き合いいただいたことは勿論、
斉藤氏が著書「信用の新世紀 ブロックチェーン後の未来」\cite{saito2017}の中で描かれるこの社会の未来像は
この研究の大きなモチベーションになりました。
彼のデジタルマネーに関する思想に出会わなければ、
本研究における問題意識を抱くこともなく社会契約を研究対象として選ぶこともなかったと言えます。
また、代々KGLとしてNECOの運営を担ってくださった阿部涼介氏、菅藤 佑太氏、島津 翔太氏、渡邉 聡紀氏にも感謝いたします。
特に慶應義塾大学 政策・メディア研究科 博士課程の阿部涼介氏は自分にも他人にも厳しい方でしたが、
加えて、彼の研究への姿勢から多くを学ぶことができました。
日頃より、RGの運営や講義、研究発表でお世話になっている
慶應義塾大学教授 村井純博士、同学部教授 中村修博士、同学部教授 楠本博之博士、同学部教授 高汐一紀博士、
同学部教授 Rodney D.Van Meter III 博士、同学部准教授 植原啓介博士、
同学部教授 三次仁博士、同学部教授 中澤仁博士、同学部教授 武田圭史博士、
同大学政策・メディア研究 科特任准教授 佐藤雅明博士、同大学政策・メディア研究科特任教授 鈴木茂哉博士に感謝いたします。
そうした運営に携わってくださったRG Coordinatorの方々もありがとうございます。
最後に、慶應SFCという素敵なキャンパスでの大学生活を支えてくださった
教員の皆さまや友人、家族に感謝しております。

\chapter{問題提起}
\section{本論の問題提起}
これまでのゲーム理論を用いて社会契約のメカニズムを説明しようとする研究では、
強制執行力の存在が仮定されている、もしくは外部の強制執行力の存在が仮定されている上で議論が進められていた。
外部の強制執行力が存在している場合については、
別の社会契約が成立している必要があるため、その別の社会契約が成立しているメカニズムを説明する必要が生じ問題が堂々巡りに陥る。
一方、外部の強制執行力が存在しない場合については、
社会の過去の歴史が決定する社会指標を利用することで社会契約が成立するとされている(Binmore 2005\cite{binmore2005})が、
この研究は還元主義的であり具体的に成員達がどのような振る舞いをすることで、そうした歴史が決定されどのように社会指標を導き出すのかを決定することは困難であった。

そこで本研究では、改めて外部の強制執行力が存在しない場合について、合意された約束を成員達に遵守させることが可能なのかという問題に取り組み、
成員達のどのような振る舞いによって、歴史を決定し、社会指標を導き出せばよいのかを計算可能なレベルで明確にする。


\section{問題解決の要件}
先の問題が解決されるためには、成員達が合意された約束を遵守する状態を作り出せていることを示す他に、
下記の3つの要件が満たされている必要がある。

\subsection{歴史を定義する}
第1に、社会契約の成立のために必要な歴史とはいかなるものかを定義する必要がある。
これは社会指標を計算するために各成員が記録すべき過去のある集団の何らかの状態である。


\subsection{社会指標の計算方法を定義する}
第2に、歴史から社会指標を計算する方法を定義する必要がある。\cite{simon:1947}
合意された約束が遵守されるようにするため、
この社会指標は約束を履行する成員の社会指標が上がり、
約束を保護にする成員の社会指標が下がるように設計する必要がある。


\subsection{成員の振る舞いのみで記述できる}
第3に、各成員の振る舞いによってのみ、歴史が共有されて各成員の社会指標を計算でき、その振る舞いを記述可能である必要がある。
これは外部の強制執行力が存在しない場合、あらゆる歴史も計算された社会指標も、
その正当性を保証した状態で外部に記録共有することができないためである。


\chapter{仮説と検証方法}
\section{本論の仮説}
外部の強制執行力が存在しない場合であっても、
集団を構成する成員の性質によっては強制執行力を生み出すことが可能であり、
それによって成員たちに合意した約束を遵守させることが可能である。


\section{仮説の検証方法}
本論では、3つの補題の仮説検証を行うことで、先の仮説を示す。

第1に、全ての成員が完全に合理的である場合、
成員の振る舞いによって決定された歴史から各成員の社会指標を算出できる外部の強制執行力が、
成員達に約束を遵守させるようなインセンティブ設計をできないことを示す。
私達は、この検証のために、任意の約束を商取引契約として結ぶ売り手と買い手の商取引ついて考える。
具体的には、売り手が報告した商取引の結果を歴史として記録し、各成員の社会指標である通貨保有量を決定することができる
外部の強制執行力としての「商取引システム」の存在を仮定する。
また、この「商取引システム」を用いて行われる商取引を「商取引ゲーム」という非協力戦略型ゲームとしてモデリングする。
そして、買い手と売り手の各戦略の利得を比較することで、
全ての成員が合理的である場合に「商取引システム」から報告された商取引の結果に基づいて不正が防止されるような買い手と売り手の通貨保有量を決定することが不可能であることを示す。

第2に、一部あるいは全部の成員が限定合理的である場合、
成員の振る舞いによって決定された歴史から各成員の社会指標を算出できる外部の強制執行力が、
合理的な成員達に約束を遵守させるようなインセンティブ設計をできることを示す。
この検証のために、我々は先の「商取引ゲーム」において報復的な戦略(買い手が商取引契約を履行しなかった場合に「失敗」を報告する戦略)をとる成員のみによって行われる
「倫理ある商取引ゲーム」について考える。
はじめの検証と同じように、この「倫理ある商取引ゲーム」における買い手と売り手の各戦略の利得を比較することで、
このモデルにおいては「商取引システム」から報告された商取引の結果に基づいて不正を防止できるような買い手と売り手の通貨保有量を決定することが可能であることを示す。
また、この条件に基づいて各誠意の通貨保有量を操作する商取引システムを設計し、
本来の「商取引ゲーム」に適用してマルチエージェントシミュレーションを用いた実験を行うことで、
全ての成員が先に述べた報復的な戦略を取る場合でなくても、一部あるいは全部の成員が報復的な戦略をとっていれば、不正が防止されうることを示す。

第3に、外部の強制執行力が存在しない場合でも、一部あるいは全部の成員が限定合理的である場合、
各成員の振る舞いのみによって成員達に約束を遵守させる強制執行力を生じさせることが可能であることを示す。
この検証のために、我々は先の検証で外部の執行力として存在が仮定されていた「商取引システム」を
集団が構成する複雑系の中で各誠意の振る舞いによって自己組織化されたシステムとして設計し、第2の検証と同様の実験を行う。
「商取引システム」を各成員の振る舞いによって自己組織化されたシステムとして再現することができ、
一部あるいは全部の成員が報復的な戦略をとっていれば、不正が防止されうることを示す。

これら3つの補題の検証により、外部の強制執行力が存在しない場合であっても、
集団を構成する成員の性質によっては強制執行力を生み出すことが可能であり、
それによって成員たちに合意した約束を遵守させることが可能だとわかる。
また、問題提起の要件で述べた社会契約の成立に必要となる歴史の定義や社会指標の計算方法、
具体的な成員の振る舞いがそれぞれ明確になる、

\chapter*{謝辞}
本研究を進めるにあたり、2016年度秋学期から所属していた研究軍団NECO及び
RG(村井・徳田・中村・楠本・高汐・バンミーター・植原・三次・中澤・手塚・武田合同研究会)の皆様、
慶應SFCの教員の皆様、大学進学を支援してくださった家族に感謝します。
中でも早稲田大学 大学院経営管理研究科 教授 斉藤賢爾氏には大変お世話になりました。
本研究が全く問題意識をはっきりとしない頃から議論にお付き合いいただいたことは勿論、
斉藤氏が著書「信用の新世紀 ブロックチェーン後の未来」\cite{saito2017}の中で描かれるこの社会の未来像は
この研究の大きなモチベーションになりました。
彼のデジタルマネーに関する思想に出会わなければ、
本研究における問題意識を抱くこともなく社会契約を研究対象として選ぶこともなかったと言えます。
また、代々KGLとしてNECOの運営を担ってくださった阿部涼介氏、菅藤 佑太氏、島津 翔太氏、渡邉 聡紀氏にも感謝いたします。
特に慶應義塾大学 政策・メディア研究科 博士課程の阿部涼介氏は自分にも他人にも厳しい方でしたが、
加えて、彼の研究への姿勢から多くを学ぶことができました。
日頃より、RGの運営や講義、研究発表でお世話になっている
慶應義塾大学教授 村井純博士、同学部教授 中村修博士、同学部教授 楠本博之博士、同学部教授 高汐一紀博士、
同学部教授 Rodney D.Van Meter III 博士、同学部准教授 植原啓介博士、
同学部教授 三次仁博士、同学部教授 中澤仁博士、同学部教授 武田圭史博士、
同大学政策・メディア研究 科特任准教授 佐藤雅明博士、同大学政策・メディア研究科特任教授 鈴木茂哉博士に感謝いたします。
そうした運営に携わってくださったRG Coordinatorの方々もありがとうございます。
最後に、慶應SFCという素敵なキャンパスでの大学生活を支えてくださった
教員の皆さまや友人、家族に感謝しております。

\chapter*{謝辞}
本研究を進めるにあたり、2016年度秋学期から所属していた研究軍団NECO及び
RG(村井・徳田・中村・楠本・高汐・バンミーター・植原・三次・中澤・手塚・武田合同研究会)の皆様、
慶應SFCの教員の皆様、大学進学を支援してくださった家族に感謝します。
中でも早稲田大学 大学院経営管理研究科 教授 斉藤賢爾氏には大変お世話になりました。
本研究が全く問題意識をはっきりとしない頃から議論にお付き合いいただいたことは勿論、
斉藤氏が著書「信用の新世紀 ブロックチェーン後の未来」\cite{saito2017}の中で描かれるこの社会の未来像は
この研究の大きなモチベーションになりました。
彼のデジタルマネーに関する思想に出会わなければ、
本研究における問題意識を抱くこともなく社会契約を研究対象として選ぶこともなかったと言えます。
また、代々KGLとしてNECOの運営を担ってくださった阿部涼介氏、菅藤 佑太氏、島津 翔太氏、渡邉 聡紀氏にも感謝いたします。
特に慶應義塾大学 政策・メディア研究科 博士課程の阿部涼介氏は自分にも他人にも厳しい方でしたが、
加えて、彼の研究への姿勢から多くを学ぶことができました。
日頃より、RGの運営や講義、研究発表でお世話になっている
慶應義塾大学教授 村井純博士、同学部教授 中村修博士、同学部教授 楠本博之博士、同学部教授 高汐一紀博士、
同学部教授 Rodney D.Van Meter III 博士、同学部准教授 植原啓介博士、
同学部教授 三次仁博士、同学部教授 中澤仁博士、同学部教授 武田圭史博士、
同大学政策・メディア研究 科特任准教授 佐藤雅明博士、同大学政策・メディア研究科特任教授 鈴木茂哉博士に感謝いたします。
そうした運営に携わってくださったRG Coordinatorの方々もありがとうございます。
最後に、慶應SFCという素敵なキャンパスでの大学生活を支えてくださった
教員の皆さまや友人、家族に感謝しております。

\chapter*{謝辞}
本研究を進めるにあたり、2016年度秋学期から所属していた研究軍団NECO及び
RG(村井・徳田・中村・楠本・高汐・バンミーター・植原・三次・中澤・手塚・武田合同研究会)の皆様、
慶應SFCの教員の皆様、大学進学を支援してくださった家族に感謝します。
中でも早稲田大学 大学院経営管理研究科 教授 斉藤賢爾氏には大変お世話になりました。
本研究が全く問題意識をはっきりとしない頃から議論にお付き合いいただいたことは勿論、
斉藤氏が著書「信用の新世紀 ブロックチェーン後の未来」\cite{saito2017}の中で描かれるこの社会の未来像は
この研究の大きなモチベーションになりました。
彼のデジタルマネーに関する思想に出会わなければ、
本研究における問題意識を抱くこともなく社会契約を研究対象として選ぶこともなかったと言えます。
また、代々KGLとしてNECOの運営を担ってくださった阿部涼介氏、菅藤 佑太氏、島津 翔太氏、渡邉 聡紀氏にも感謝いたします。
特に慶應義塾大学 政策・メディア研究科 博士課程の阿部涼介氏は自分にも他人にも厳しい方でしたが、
加えて、彼の研究への姿勢から多くを学ぶことができました。
日頃より、RGの運営や講義、研究発表でお世話になっている
慶應義塾大学教授 村井純博士、同学部教授 中村修博士、同学部教授 楠本博之博士、同学部教授 高汐一紀博士、
同学部教授 Rodney D.Van Meter III 博士、同学部准教授 植原啓介博士、
同学部教授 三次仁博士、同学部教授 中澤仁博士、同学部教授 武田圭史博士、
同大学政策・メディア研究 科特任准教授 佐藤雅明博士、同大学政策・メディア研究科特任教授 鈴木茂哉博士に感謝いたします。
そうした運営に携わってくださったRG Coordinatorの方々もありがとうございます。
最後に、慶應SFCという素敵なキャンパスでの大学生活を支えてくださった
教員の皆さまや友人、家族に感謝しております。

\chapter*{謝辞}
本研究を進めるにあたり、2016年度秋学期から所属していた研究軍団NECO及び
RG(村井・徳田・中村・楠本・高汐・バンミーター・植原・三次・中澤・手塚・武田合同研究会)の皆様、
慶應SFCの教員の皆様、大学進学を支援してくださった家族に感謝します。
中でも早稲田大学 大学院経営管理研究科 教授 斉藤賢爾氏には大変お世話になりました。
本研究が全く問題意識をはっきりとしない頃から議論にお付き合いいただいたことは勿論、
斉藤氏が著書「信用の新世紀 ブロックチェーン後の未来」\cite{saito2017}の中で描かれるこの社会の未来像は
この研究の大きなモチベーションになりました。
彼のデジタルマネーに関する思想に出会わなければ、
本研究における問題意識を抱くこともなく社会契約を研究対象として選ぶこともなかったと言えます。
また、代々KGLとしてNECOの運営を担ってくださった阿部涼介氏、菅藤 佑太氏、島津 翔太氏、渡邉 聡紀氏にも感謝いたします。
特に慶應義塾大学 政策・メディア研究科 博士課程の阿部涼介氏は自分にも他人にも厳しい方でしたが、
加えて、彼の研究への姿勢から多くを学ぶことができました。
日頃より、RGの運営や講義、研究発表でお世話になっている
慶應義塾大学教授 村井純博士、同学部教授 中村修博士、同学部教授 楠本博之博士、同学部教授 高汐一紀博士、
同学部教授 Rodney D.Van Meter III 博士、同学部准教授 植原啓介博士、
同学部教授 三次仁博士、同学部教授 中澤仁博士、同学部教授 武田圭史博士、
同大学政策・メディア研究 科特任准教授 佐藤雅明博士、同大学政策・メディア研究科特任教授 鈴木茂哉博士に感謝いたします。
そうした運営に携わってくださったRG Coordinatorの方々もありがとうございます。
最後に、慶應SFCという素敵なキャンパスでの大学生活を支えてくださった
教員の皆さまや友人、家族に感謝しております。

\chapter*{謝辞}

\bibliographystyle{unsrt}
\bibliography{./bibliography}

\end{document}
