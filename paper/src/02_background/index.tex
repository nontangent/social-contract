\chapter{背景}
\section{社会契約とは}
社会契約とは、ある集団において、「各成員に合意された約束の履行を強制する力」を生じさせるプロセスである。

\section{スタグハントゲーム}
\begin{table*}[htbp]
  \centering
  \begin{tabular}{|l|l|l|l|l|l|}
  \hline
  \multicolumn{2}{|l|}{\multirow{2}{*}{}} & \multicolumn{2}{l|}{$hunter_2$} \\ \cline{3-4}
  \multicolumn{2}{|l|}{                 } & 鹿 & 野兎 \\ \hline
  \multirow{2}{*}{$hunter_1$}
  & 鹿 &(2,2)&(0,1)\\ \cline{2-4}
  & 野兎 &(1,0)&(1,1)\\ \hline
  \end{tabular}
  \caption{スタグハントゲームの利得表}
  \label{stag-hunt-game}
\end{table*}
スタグハントゲームとは、ルソーの「人類不平等起源論」\cite{rousseau1999discourse}に登場する「鹿狩りの寓話」をモデリングした非協力戦略型ゲームである。\cite{skyrms2001}
二人のハンターが協力して鹿を狩るか、相手を裏切って野兎を狩るかを選択するが、
鹿は二人で協力しなければ狩ることができず、1人だけで狩ろうとすると何も得ることができない。
このゲームは代表的な囚人のジレンマゲームであり、

\section{これまでの社会契約の議論}
デイビッド・ヒュームは

\section{フォーク定理}
無限回の繰り返し囚人のジレンマゲームにおいて、協力解が均衡解になるという定理。

\section{強制執行力}
強制執行力とは、合意された利得の分配を強制的に執り行っている力である。
先に紹介したスタグハントゲームでいうところの、
二人で協力して鹿を狩った場合にその鹿を二人で分け合うという取り決めを守らせている力である。
これまでの社会契約の議論では、こうした強制執行力の存在が暗黙的に仮定されていた。
一例としては、ロールズの「自然の義務」、ハーサニの「道徳的コミットメント」が挙げられる。

\section{外部の強制執行力}
強制執行力のうち、集団の外部の機関によってもたらされる力を外部の強制執行力と呼ぶ。
外部の強制執行力が存在する場合、そこには別の社会契約が存在していることになる。
そのため、社会契約によってx

相互監視によって成り立つとされている。


\section{限定合理性}
限定合理性とは、意思決定主体が認知能力の限界によって限定された合理性しか発揮することができない性質である。
ゲーム理論においても、「合理的で利己的な経済人」という仮定の上では説明が困難だった実社会での協力的な行動の説明に用いられている。
最後通牒ゲームの実験は、その最もたる例であり、実社会のプレイヤーは相手よりも少ない取り分を提示されると自身の利得が0になるにも関わらず報復的な戦略を選んでいた。\cite{GUTH1982367}

\section{複雑系}
システムの構成要素の振る舞いによって全体の
本論においては、社会契約を成立させようとする集団を、成員によって構成される複雑系として捉える。

\section{自己組織化}
自己組織化とは、複雑系において、システム全体を俯瞰できない構成要素の振る舞いによって、全体として秩序だった振る舞いがなされることである。
本論においては、3つ目の検証で、外部に存在を仮定してた強制執行力を用いる。

\section{ビザンチン将軍問題}
ビザンチン将軍問題とは、〜である。\cite{lamport1982}

\begin{itemize}
  \item[IC1.] すべての誠実な副官は同じ司令に従う。
  \item[IC2.] 司令官が誠実な場合、全ての誠実な副官は彼の送った司令に従う。
\end{itemize}

\section{署名されたメッセージによる解決策}
A1〜A5の5つの仮定の上で、下記のアルゴリズムによって、$m+1$人以上の将軍がいる場合にIC1とIC2が満たされる。

\subsection{仮定}
\begin{itemize}
  \item[A1] 送信されたすべてのメッセージは正しく到達する
  \item[A2] メッセージの受信者は誰が送信したのかわかる
  \item[A3] メッセージが届かないことを検知できる
  \item[A4] 誠実な将軍の署名は偽造できず、署名されたメッセージの内容が変更されても、それを検知することができる。
  \item[A5] 誰でも将軍の署名の信憑性を検証することができる。
\end{itemize}

\subsection{関数 $choice(V)$}


\subsection{アルゴリズム $SM(m)$}
\begin{enumerate}
  \item $V_i=\emptyset$として初期化する。($\emptyset$は空集合)
  \item 司令官は彼の値を全ての副官に署名して送る。
  \item 各$i$について、 
  \begin{enumerate}
    \item もし$lineutenant_i$が$v:0$という形式のメッセージを受け取り、まだ何の命令も受けていない場合は、
    \begin{enumerate}
      \item $lineutenant_i$は$ V_i$を${v}$にする。 
      \item $lineutenant_i$は他のすべての中尉にメッセージ$v:0:i$を送ります。
    \end{enumerate}
    \item もし$lieutenant_i$が$v:0:j_1:...:j_k$という形式のメッセージを受け取り、$v$が集合$V_i$に入っていない場合は
    \begin{enumerate}
      \item $lieutenant_i$は$v$を$V_i$に追加する。
      \item もし$k<m$であれば、$lieutenant_i$は$j_1, ..., j_k$以外のすべての副官に$v:0:j_1:...:j_k:i$というメッセージを送る。
    \end{enumerate}
  \end{enumerate}
  \item 各$i$について。$lineutenant_i$がこれ以上メッセージを受け取らない場合、$lieutenant_i$は命令$choice(V_i)$に従う。
\end{enumerate}

\section{マルチエージェントシミュレーション}
マルチエージェントシミュレーションとは、
与えられた方策に従って振る舞う複数のエージェントと環境を定義し、
計算機によってそれらのエージェントの振る舞いをシミュレーションすることで、
エージェント達の振る舞いによってもたらされる相互作用を観察する研究手法である。
本論では、分散システムとして設計されたシステムの検証のために用いる。

