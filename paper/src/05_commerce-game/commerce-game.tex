
% また,この商取引のモデルは,第3ステップ以降の$seller$の行動選択と,それに対する第4ステップの$ buyer$の行動選択を,非協力戦略型ゲームとしてとらえられる.ここで,戦略$ s^{player}_{n}$を$ player$(ここでは$seller$か$buyer$)の取りうる戦略番号$n$の戦略として,$ seller$と$ buyer$のそれぞれの戦略は以下のように定義する.\\



% また,商取引終了時の$ player$の保有する通貨量の変化によって生じる利得を,第4ステップでの報告が「成功」だった場合は$ r^{player}_{success}$,「失敗」だった場合は$ r^{player}_{failure}$とし,商品の所有によって生じる利得を$ goods$と表す.
% ここで,$ seller$と$ buyer$の任意の戦略組$ (s^{seller}_n, s^{buyer}_n)$の際の$ seller$と$ buyer$の各利得は表\ref{gametable}のようになる.


