% Computable Social Contract as a Complex System and Analysis it's Chaos

\chapter{はじめに}

  \section{なぜ私たちは法を守るのか?}
  この問いに答えることが社会契約という概念が研究される主な目的である。
  普段、我々がこの問いの答えを意識せずに暮らしているが、考えてみれば説明するのは非常に難しい問題である。
  例えば、あなたが商店でホットドッグを買ったとしよう。
  日常的に考えれば、代金を払い、ホットドッグを受け取る。それだけの話である。
  だが、この約束はなぜ守られるのだろうか?仮にあなたが代金を払った後、
  店主が何事もなかったかのように振る舞い、一向にホットドッグを渡さなかったとする。
  あなたは警察に通報すれば必ず解決するといえるだろうか?
  
  治安の良い環境で生きている方にとっては理解できない話かもしれないが、
  新宿の歌舞伎町やブラジルのリオデジャネイロでは、そうしたことは日常茶飯事ある。
  悪ければ国家ぐるみで犯罪は起きている。
  泣き寝入りすることは本当にない?

  要はこの問に答えるのは難しいのである。
  この問に答えるためには、一切の外部の強制執行力に頼ることなく、
  法が強制力を持つ仕組みを説明できなければならない。

  \section{これまでの社会契約の議論}
  自然状態と社会状態。


  現実の社会を見て分かる通り、万人が法を守るわけではないし、万人が法を守らないわけでもない。
  自然状態も社会状態も仮定の話である。

  近代において、ホッブズやロック、ルソーによって議論された。
  これは、国家の正当性を議論するものであり、王権神授説への反論となり、フランス革命などの理論的な正当性となった。

  現代では、ジョン・ロールズやハーサニーによって、正義論の中で議論されている。
  これらは伝統的な社会契約を一般化し、そのから導き出される公正な法とはどういったものかについて議論している。

  また、ゲーム理論や進化ゲーム理論を用いて、
  社会契約のプロセス自体を説明するには至っていない。


  \section{インターネットと社会契約} %研究の動機 なぜこの研究が必要なのか?
  インターネットの登場で社会は劇的に変化した。
  国境を意識せずに、ユーザーは他国のサービスにアクセスし、多国籍企業も増えた。
  また、ブロックチェーンのような
  インターネットは国家を超越した地球のインフラとして機能しだしている。
  これは「地球規模OS」のような概念が現実味を帯びてくる。
  ここで問題となるのはインターネット上のサービスの正当性を誰が保証するのかという問題である。
  これまではサービスの正当性は国家の法に縛られた企業が保証していればよかった。
  これからは、インターネットユーザーこそが国民であり、そこで生じる法によってサービスの正当性が保証される必要がある。
  そこで、今一度、社会契約について考えるときである。  


  \section{本論における社会契約の定義}
  本論における社会契約とは、ある集団において個々がある取り決めに合意することによって、
  その取り決めの遵守を強制させるプロセスである。


  \section{本論の目的}
  本論の目的は社会契約を一般化した計算可能なモデルを構築し、
  コンピューターシミュレーションを用いて、
  社会の成員の性質と社会契約の成立の関係性を明らかにすることである。

  先に述べたように、我々の見据えるインターネット上での社会契約において、
  人々はコンピューターを用いて、それを成すことになる。
  ならば社会契約のモデルは計算可能なレベルで一般化される必要があるだろう。
  また、はじめに述べたように、我々の生きる社会は完全な無法地帯でなければ、全員が法に従っているわけでもない。
  中には、法に反した行動をとりながらも、
  モデルが完成しただけでは実用にはいたれない、そのシステムがいかなる場合にどれだけ安全であるのかを知る必要がある。
  そのため、モデルを構築するとともに、コンピューターシミュレーションによってその安全性を検証する。

  \section{本論のアプローチ}
  国家が法と通貨で規定されるならば、それは約束と信用の積み重なりである。
  商取引を通貨と約束のやりとりと考え、社会契約とは複数の商取引が絡み合う複雑系によるものではないかと考える。


  \section{本論の構成}
  3章では商取引をモデリングした「商取引ゲーム」についてゲーム理論を用いて分析し、
  インセンティブ設計によって不正を防止することが困難であることを示す。
  

  4章では、
  商取引に失敗した場合に、その結果を正しくシステムに報告するプレイヤーを誠実なプレイヤーであると考え、
  この限定合理性を仮定した「倫理ある商取引ゲーム」をモデリングしシミュレーションすることで、
  参加するプレイヤーの性質によっては不正を防止することが可能であることを示す。

  5章では、
  商取引システムを運営する約束を、商取引ゲームで取引されていた商品(サービス)とすることで、
  外部の強制執行力として存在していた商取引システムを自己組織化させる。
  これによって、外部の強制執行力に依存することなく、
  社会の成員たちによって構成された複雑系の内部で商取引システムが稼働し、
  「倫理ある商取引ゲーム」と同様に参加するプレイヤーの性質によって不正が起きなくなることを示す。
  
  これによって、あらゆる強制力に依存されることなく、
  約束が複雑系の中で強制力を持ち、法として機能することがわかる。

