\section{評価}

\begin{figure}[h]
  \begin{tabular}{cc}
    %---- 最初の図 ---------------------------
    \begin{minipage}[t]{1\hsize}
      \centering
      % \includegraphics[keepaspectratio, width=1\linewidth]{ethical-aggregate.png}
      \caption{「倫理ある商取引ゲーム」}
      \label{ethical-experiment-aggregate}
    \end{minipage} \\
    %---- 2番目の図 --------------------------
    \begin{minipage}[t]{1\hsize}
      \centering
      % \includegraphics[keepaspectratio, width=1\linewidth]{no-ethical-aggregate.png}
      \caption{「商取引ゲーム」}
      \label{non-ethical-experiment-aggregate}
    \end{minipage}
    % ---- 図はここまで ----------------------
  \end{tabular}
\end{figure}

\subsection{「倫理ある商取引ゲーム」での評価}
「倫理ある商取引ゲーム」における286回の実験データを誠実なエージェントの数ごとに集計し,「商取引ゲーム」を10000万回を繰り返したときの真の成功率の平均をプロットしたのがFigure\ref{ethical-experiment-aggregate}である.誠実なエージェントが3機以下の場合は真の商取引の成功率の平均はいずれも0\%であったが,5機以上の場合はいずれも100\%に収束していた.

\subsection{「商取引ゲーム」での評価}
同様に,通常の「商取引ゲーム」における1000件の実験データを誠実なエージェントの数ごとに集計し,「商取引ゲーム」を10000万回を繰り返したときの真の成功率の平均をプロットしたのがFigure\ref{non-ethical-experiment-aggregate}である.こちらも誠実なエージェントが5機以上の場合は真の商取引の成功率の平均はいづれも99\%を超えていた.また,「倫理ある商取引ゲーム」での実験に比べて4機の場合に比べてこちらの方が真の商取引の成功率の平均は高かく,3機以下の場合でも完全に0\%にはならなかった.
