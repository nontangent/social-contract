\section{考察}
「倫理ある商取引ゲーム」において誠実なエージェントが1体以下の場合,どの2体のエージェントで「商取引ゲーム」を行っても商取引は失敗する.そのため誠実なエージェントが1体以下の場合に商取引の成功率が0\%になるのは納得である.しかし,今回の実験では2体もしくは3体の場合でも真の商取引の成功率の平均が0\%になっている.この点は疑問である.また,今回のインセンティブ設計のアルゴリズムは「倫理ある商取引ゲーム」を前提としているにも関わらず,通常の「商取引ゲーム」の方が誠実なエージェントが4体のときの真の商取引の成功率の平均値が高いのは不可解である.これらの原因としては,確率的な試行であるにも関わらず「倫理ある商取引ゲーム」においては286パターンのエージェントの構成で1回づつしかサンプリングを行っていないことや,エージェントの構成が19960パターン考えられる通常の「商取引ゲーム」で1000回しか試行をしていないことなどが原因として考えられる.また,最低信頼度$T^{player}$と誠実さ$P^{player}_1$の関係性が$p^{buyer}_1$との関係性とすり替わっていることも原因として考えられる.実験方法やインセンティブ設計を修正する必要があると思われる.現状の結果から唯一いえることがあるとすれば,誠実なエージェントの割合が一定を超えると「商取引ゲーム」の不正行為が防止できるということである.
