\section{倫理ある商取引システムの詳細}
「倫理ある商取引ゲーム」においては,最低信頼度$ T^{player} $を用いた上記の条件式を満たせば,不正を防止することが可能になる.ただ,上記の条件のみでは「商取引システム」から保有通貨の変化量の組$ (r^{seller}_{success}, r^{seller}_{failure}, r^{buyer}_{success}, r^{buyer}_{failure}) $を一意に決定できないので,いくつかの条件を追加する.


\subsection{取引成功前後の残高の変化}
商取引成功前後では$ buyer $は商品$ goods $の価格$ price $だけ残高が減り$ seller $は$ price $だけ残高が増えるものとする.そのため商取引前後では$ seller $と$ buyer $の残高の合計は変化しない.ここから,$ r^{seller}_{success} $と$ r^{buyer}_{success} $は以下のように記せる.

$ r^{seller}_{success} = price $
$ r^{buyer}_{success} = -price $
$ r^{seller}_{success} + r^{buyer}_{succcess} = 0 $

\subsection{エスクロー係数 E}
まずは「失敗」が報告された時に$ seller $と$ buyer $から失われる通貨の量の合計を$ EscrowCost $とおいて考え,同時に商品価格$ price $にエスクロー係数$ E $を掛けたものとする.(ここで$ price $は$ goods $の価格である)

$ EscrowCost \equiv (r^{buyer}_{success} - r^{buyer}_{failure}) + (r^{seller}_{success} - r^{seller}_{failure}) $
$ = E \cdot price $

\subsection{負担比率}
ここで問題となるのは$ EscrowCost $を求めるための$ seller $と$ buyer $の負担比率をどのようにして決定するかである.本稿では最低信頼度$ T^{player} $を用いてこの負担比率を決定することとする.

$ (r^{buyer}_{success} - r^{buyer}_{failure}):(r^{seller}_{success} - r^{seller}_{failure}) = w(T^{buyer}):w(T^{seller}) $

ここで最低信頼度が高い$ player $の方が負担比率が小さくなるように責任比重関数$ w(x) $は値域$ 0 \leq  x \leq 1 $において$ w'(x)>0 $を満たすものとする.

\subsection{責任比重関数$ w(x) $}
本稿では負担比重関数は恒等写像$ w(x)=x $とする.

\subsection{ReputationWeights}
最低信頼度$ T^{player} $を求めるにあたって$ r.w.^{opportunity} $を決定する必要がある.$ r.w.^{opportunity} $は誠実さ$ T^{player}_1 $を求める際に$ ReportedSuccessRate(player, opportunity) $に係る任意の重みである.これはつまり任意の$ player $が誠実な戦略をとる主観的確率を考える際に,$ player $と$ opportunity $の間であった過去の商取引での報告された成功率をどの程度信じるかである.仮に$ player $と$ opportunity $の間で信頼関係があれば$ ReportedSuccessRate $は$ 1 $になる.なので「商取引システム」の参加者数$ n $が特定できる場合は$ \frac{1}{n} $が妥当だろう.逆に不特定多数であり,同一の意志によって複数のプレイヤーが動いている場合は,保有している通貨量$ b^{opportunity} $の全体の通貨の総量に占める割合が良いだろう.

$ r.w.^{opportunity} \equiv \frac{b^{opportunity}}{\sum^{players}_{i}b^{i}} $


\subsection{EscrowCostの分配}
「失敗」が報告されたときに失われる$ EscrowCost $は消滅するのではなく参加者に分配する.これは全体の通貨量の減少によって通貨の価値が上がって商品価格が下がるという経済原理が生じ,インセンティブ設計が複雑化するのを防ぐためである.そこで$ EscrowCost $として失われた通貨は任意の重み$ e.w.^{player} (\sum^{players}_{player}e.w.^{player} = 1) $を用いて分配し「商取引ゲーム」の前後で全体の通貨量を等しくする.

\subsection{EscrowWeights}
EscrowCostの分配に関しては「商取引システム」に参加している各プレイヤーの通貨の保有率に応じて分配する.$ seller $と$ buyer $を含んでいるのは,仮に$ seller $と$ buyer $以外のすべてのプレイヤーで分配した場合,$ seller $もしくは$ buyer $は結託する別のプレイヤーに保有する通貨の一部を一時的に預けることで,その分だけ$ EscrowCost $の負担を軽減することが可能になるためである.そこで本稿では任意の重み$ e.w.^{player} $を以下のように定義する.

$ e.w.^{player} \equiv \frac{b^{player}}{\sum^{players}_{escrow}b^{escrow}} $

\subsection{謎の条件}
$ \frac{w(T^{buyer}_1)E \cdot price}{w(T^{buyer}_1) + w(T^{seller}_1)} \geq \frac{price}{T^{buyer}} \geq \frac{goods}{p^{buyer}_1} $

上記の条件式から$ \frac{price}{ T^{buyer} } $でうまくいくはずだったが何故かうまく行かず,$ \frac{price}{ \min(T^{buyer}, T^{seller})} $をもちいたらうまくいったのでこちらを採用することとした.$ p^{buyer} $と$ P^{player}_1 $の関係性に問題があるためだと思われる.

$ \frac{w(T^{buyer}_1)E \cdot price}{w(T^{buyer}_1) + w(T^{seller}_1)} \geq \frac{price}{ \min(T^{buyer}, T^{seller})} $


\subsection{残高の変化量の組$ (r^{seller}_{success}, r^{seller}_{failure}, r^{buyer}_{success}, r^{buyer}_{failure}) $}
上記の条件群を用いて残高の変化量の組$ (r^{seller}_{success}, r^{seller}_{failure}, r^{buyer}_{success}, r^{buyer}_{failure}) $を決定する.


$ r^{seller}_{success}+r^{buyer}_{success} = 0 $

$ r^{seller}_{failure}+r^{buyer}_{failure} = -E \cdot price $

$ r^{seller}_{success} - r^{seller}_{failure} = \frac{w(T^{buyer}_1)E \cdot price}{w(T^{buyer}_1) + w(T^{seller}_1)} \geq \frac{price}{T^{buyer}} $

$ r^{buyer}_{success} - r^{buyer}_{failure} = \frac{w(T^{seller}_1)E \cdot price}{w(T^{buyer}_1) + w(T^{seller}_1)} \geq 0 $

$ \frac{w(T^{buyer})E \cdot price}{w(T^{buyer}) + w(T^{seller})} = \frac{price}{\min(T^{buyer}, T^{seller})} $

$ E $ = $ \frac{w(T^{buyer})+w(T^{seller})}{w(T^{buyer}) \cdot \min(T^{buyer}, T^{seller})} $

$ r^{seller}_{success}-r^{seller}_{failure} = \frac{price}{min(T^{buyer}, T^{seller})} $

$ r^{buyer}_{success} - r^{buyer}_{failure} = \frac{w(T^{seller}) \cdot price}{w(T^{buyer}) \cdot \min(T^{buyer}, T^{seller})} $


$ r^{seller}_{success} = price $
$ r^{buyer}_{success} = -price $
$ r^{seller}_{failure} = price \cdot (1 - \frac{1}{min(T^{buyer}, T^{seller})}) $
$ r^{buyer}_{failure} = - price \cdot (\frac{T^{seller}}{T^{buyer} \cdot \min(T^{buyer}, T^{seller})} + 1) $
